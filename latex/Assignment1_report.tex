\documentclass[11pt]{report}
\usepackage{./assignment_programming}
\usepackage{slashbox}
\usepackage{enumerate}
\usepackage[shortlabels]{enumitem}
\usepackage{stmaryrd}
\usepackage{graphicx}
\usepackage[final]{pdfpages}
\usepackage{array}
\usepackage{multirow}
%\PassOptionsToPackage{normalem}{ulem}
%\usepackage{ulem}

\RequirePackage{latexsym}
\RequirePackage{amsmath}
\RequirePackage{amssymb}
\RequirePackage{bm}
\RequirePackage{url}
\RequirePackage{algorithmic}
\RequirePackage{algorithm2e}


%%%%%%%% Stock standard definitions %%%%%%%%%%%%%%%

%\newcommand{\ab}{\mathbf{a}}
%\newcommand{\bb}{\mathbf{b}}
%\newcommand{\cbb}{\mathbf{c}}
%\newcommand{\db}{\mathbf{d}}
%\newcommand{\eb}{\mathbf{e}}
%\newcommand{\fb}{\mathbf{f}}
%\newcommand{\gb}{\mathbf{g}}
%\newcommand{\hb}{\mathbf{h}}
%\newcommand{\ib}{\mathbf{i}}
%\newcommand{\jb}{\mathbf{j}}
%\newcommand{\kb}{\mathbf{k}}
%\newcommand{\lb}{\mathbf{l}}
%\newcommand{\mb}{\mathbf{m}}
%\newcommand{\nbb}{\mathbf{n}}
%\newcommand{\ob}{\mathbf{o}}
%\newcommand{\pb}{\mathbf{p}}
%\newcommand{\qb}{\mathbf{q}}
%\newcommand{\rb}{\mathbf{r}}
%\newcommand{\sbb}{\mathbf{s}}
%\newcommand{\tb}{\mathbf{t}}
%\newcommand{\ub}{\mathbf{u}}
%\newcommand{\vb}{\mathbf{v}}
%\newcommand{\wb}{\mathbf{w}}
%\newcommand{\xb}{\mathbf{x}}
%\newcommand{\yb}{\mathbf{y}}
%\newcommand{\zb}{\mathbf{z}}
%
%\newcommand{\abtil}{\tilde{\ab}}
%\newcommand{\bbtil}{\tilde{\bb}}
%\newcommand{\cbtil}{\tilde{\cbb}}
%\newcommand{\dbtil}{\tilde{\db}}
%\newcommand{\ebtil}{\tilde{\eb}}
%\newcommand{\fbtil}{\tilde{\fb}}
%\newcommand{\gbtil}{\tilde{\gb}}
%\newcommand{\hbtil}{\tilde{\hb}}
%\newcommand{\ibtil}{\tilde{\ib}}
%\newcommand{\jbtil}{\tilde{\jb}}
%\newcommand{\kbtil}{\tilde{\kb}}
%\newcommand{\lbtil}{\tilde{\lb}}
%\newcommand{\mbtil}{\tilde{\mb}}
%\newcommand{\nbtil}{\tilde{\nbb}}
%\newcommand{\obtil}{\tilde{\ob}}
%\newcommand{\pbtil}{\tilde{\pb}}
%\newcommand{\qbtil}{\tilde{\qb}}
%\newcommand{\rbtil}{\tilde{\rb}}
%\newcommand{\sbtil}{\tilde{\sbb}}
%\newcommand{\tbtil}{\tilde{\tb}}
%\newcommand{\ubtil}{\tilde{\ub}}
%\newcommand{\vbtil}{\tilde{\vb}}
%\newcommand{\wbtil}{\tilde{\wb}}
%\newcommand{\xbtil}{\tilde{\xb}}
%\newcommand{\ybtil}{\tilde{\yb}}
%\newcommand{\zbtil}{\tilde{\zb}}
%
%
%\newcommand{\atil}{\tilde{a}}
%\newcommand{\btil}{\tilde{b}}
%\newcommand{\ctil}{\tilde{c}}
%\newcommand{\dtil}{\tilde{d}}
%\newcommand{\etil}{\tilde{e}}
%\newcommand{\ftil}{\tilde{f}}
%\newcommand{\gtil}{\tilde{g}}
%\newcommand{\htil}{\tilde{h}}
%\newcommand{\itil}{\tilde{i}}
%\newcommand{\jtil}{\tilde{j}}
%\newcommand{\ktil}{\tilde{k}}
%\newcommand{\ltil}{\tilde{l}}
%\newcommand{\mtil}{\tilde{m}}
%\newcommand{\ntil}{\tilde{n}}
%\newcommand{\otil}{\tilde{o}}
%\newcommand{\ptil}{\tilde{p}}
%\newcommand{\qtil}{\tilde{q}}
%\newcommand{\rtil}{\tilde{r}}
%\newcommand{\stil}{\tilde{s}}
%\newcommand{\ttil}{\tilde{t}}
%\newcommand{\util}{\tilde{u}}
%\newcommand{\vtil}{\tilde{v}}
%\newcommand{\wtil}{\tilde{w}}
%\newcommand{\xtil}{\tilde{x}}
%\newcommand{\ytil}{\tilde{y}}
%\newcommand{\ztil}{\tilde{z}}
%
%\newcommand{\Atil}{\tilde{A}}
%\newcommand{\Btil}{\tilde{B}}
%\newcommand{\Ctil}{\tilde{C}}
%\newcommand{\Dtil}{\tilde{D}}
%\newcommand{\Etil}{\tilde{E}}
%\newcommand{\Ftil}{\tilde{F}}
%\newcommand{\Gtil}{\tilde{G}}
%\newcommand{\Htil}{\tilde{H}}
%\newcommand{\Itil}{\tilde{I}}
%\newcommand{\Jtil}{\tilde{J}}
%\newcommand{\Ktil}{\tilde{K}}
%\newcommand{\Ltil}{\tilde{L}}
%\newcommand{\Mtil}{\tilde{M}}
%\newcommand{\Ntil}{\tilde{N}}
%\newcommand{\Otil}{\tilde{O}}
%\newcommand{\Ptil}{\tilde{P}}
%\newcommand{\Qtil}{\tilde{Q}}
%\newcommand{\Rtil}{\tilde{R}}
%\newcommand{\Stil}{\tilde{S}}
%\newcommand{\Ttil}{\tilde{T}}
%\newcommand{\Util}{\tilde{U}}
%\newcommand{\Vtil}{\tilde{V}}
%\newcommand{\Wtil}{\tilde{W}}
%\newcommand{\Xtil}{\tilde{X}}
%\newcommand{\Ytil}{\tilde{Y}}
%\newcommand{\Ztil}{\tilde{Z}}
%
%\newcommand{\abar}{\bar{a}}
%\newcommand{\bbar}{\bar{b}}
%\newcommand{\cbar}{\bar{c}}
%\newcommand{\dbar}{\bar{d}}
%\newcommand{\ebar}{\bar{e}}
%\newcommand{\fbar}{\bar{f}}
%\newcommand{\gbar}{\bar{g}}
%\newcommand{\hbr}{\bar{h}}
%\newcommand{\ibar}{\bar{i}}
%\newcommand{\jbar}{\bar{j}}
%\newcommand{\kbar}{\bar{k}}
%\newcommand{\lbar}{\bar{l}}
%\newcommand{\mbar}{\bar{m}}
%\newcommand{\nbar}{\bar{n}}
%\newcommand{\obar}{\bar{o}}
%\newcommand{\pbar}{\bar{p}}
%\newcommand{\qbar}{\bar{q}}
%\newcommand{\rbar}{\bar{r}}
%\newcommand{\sbar}{\bar{s}}
%\newcommand{\tbar}{\bar{t}}
%\newcommand{\ubar}{\bar{u}}
%\newcommand{\vbar}{\bar{v}}
%\newcommand{\wbar}{\bar{w}}
%\newcommand{\xbar}{\bar{x}}
%\newcommand{\ybar}{\bar{y}}
%\newcommand{\zbar}{\bar{z}}
%
%\newcommand{\abbar}{\bar{\ab}}
%\newcommand{\bbbar}{\bar{\bb}}
%\newcommand{\cbbar}{\bar{\cb}}
%\newcommand{\dbbar}{\bar{\db}}
%\newcommand{\ebbar}{\bar{\eb}}
%\newcommand{\fbbar}{\bar{\fb}}
%\newcommand{\gbbar}{\bar{\gb}}
%\newcommand{\hbbar}{\bar{\hb}}
%\newcommand{\ibbar}{\bar{\ib}}
%\newcommand{\jbbar}{\bar{\jb}}
%\newcommand{\kbbar}{\bar{\kb}}
%\newcommand{\lbbar}{\bar{\lb}}
%\newcommand{\mbbar}{\bar{\mb}}
%\newcommand{\nbbar}{\bar{\nbb}}
%\newcommand{\obbar}{\bar{\ob}}
%\newcommand{\pbbar}{\bar{\pb}}
%\newcommand{\qbbar}{\bar{\qb}}
%\newcommand{\rbbar}{\bar{\rb}}
%\newcommand{\sbbar}{\bar{\sbb}}
%\newcommand{\tbbar}{\bar{\tb}}
%\newcommand{\ubbar}{\bar{\ub}}
%\newcommand{\vbbar}{\bar{\vb}}
%\newcommand{\wbbar}{\bar{\wb}}
%\newcommand{\xbbar}{\bar{\xb}}
%\newcommand{\ybbar}{\bar{\yb}}
%\newcommand{\zbbar}{\bar{\zb}}
%
%\newcommand{\Ab}{\mathbf{A}}
%\newcommand{\Bb}{\mathbf{B}}
%\newcommand{\Cb}{\mathbf{C}}
%\newcommand{\Db}{\mathbf{D}}
%\newcommand{\Eb}{\mathbf{E}}
%\newcommand{\Fb}{\mathbf{F}}
%\newcommand{\Gb}{\mathbf{G}}
%\newcommand{\Hb}{\mathbf{H}}
%\newcommand{\Ib}{\mathbf{I}}
%\newcommand{\Jb}{\mathbf{J}}
%\newcommand{\Kb}{\mathbf{K}}
%\newcommand{\Lb}{\mathbf{L}}
%\newcommand{\Mb}{\mathbf{M}}
%\newcommand{\Nb}{\mathbf{N}}
%\newcommand{\Ob}{\mathbf{O}}
%\newcommand{\Pb}{\mathbf{P}}
%\newcommand{\Qb}{\mathbf{Q}}
%\newcommand{\Rb}{\mathbf{R}}
%\newcommand{\Sbb}{\mathbf{S}}
%\newcommand{\Tb}{\mathbf{T}}
%\newcommand{\Ub}{\mathbf{U}}
%\newcommand{\Vb}{\mathbf{V}}
%\newcommand{\Wb}{\mathbf{W}}
%\newcommand{\Xb}{\mathbf{X}}
%\newcommand{\Yb}{\mathbf{Y}}
%\newcommand{\Zb}{\mathbf{Z}}
%
%\newcommand{\Abtil}{\tilde{\Ab}}
%\newcommand{\Bbtil}{\tilde{\Bb}}
%\newcommand{\Cbtil}{\tilde{\Cb}}
%\newcommand{\Dbtil}{\tilde{\Db}}
%\newcommand{\Ebtil}{\tilde{\Eb}}
%\newcommand{\Fbtil}{\tilde{\Fb}}
%\newcommand{\Gbtil}{\tilde{\Gb}}
%\newcommand{\Hbtil}{\tilde{\Hb}}
%\newcommand{\Ibtil}{\tilde{\Ib}}
%\newcommand{\Jbtil}{\tilde{\Jb}}
%\newcommand{\Kbtil}{\tilde{\Kb}}
%\newcommand{\Lbtil}{\tilde{\Lb}}
%\newcommand{\Mbtil}{\tilde{\Mb}}
%\newcommand{\Nbtil}{\tilde{\Nb}}
%\newcommand{\Obtil}{\tilde{\Ob}}
%\newcommand{\Pbtil}{\tilde{\Pb}}
%\newcommand{\Qbtil}{\tilde{\Qb}}
%\newcommand{\Rbtil}{\tilde{\Rb}}
%\newcommand{\Sbtil}{\tilde{\Sbb}}
%\newcommand{\Tbtil}{\tilde{\Tb}}
%\newcommand{\Ubtil}{\tilde{\Ub}}
%\newcommand{\Vbtil}{\tilde{\Vb}}
%\newcommand{\Wbtil}{\tilde{\Wb}}
%\newcommand{\Xbtil}{\tilde{\Xb}}
%\newcommand{\Ybtil}{\tilde{\Yb}}
%\newcommand{\Zbtil}{\tilde{\Zb}}
%
%\newcommand{\Abar}{\bar{A}}
%\newcommand{\Bbar}{\bar{B}}
%\newcommand{\Cbar}{\bar{C}}
%\newcommand{\Dbar}{\bar{D}}
%\newcommand{\Ebar}{\bar{E}}
%\newcommand{\Fbar}{\bar{F}}
%\newcommand{\Gbar}{\bar{G}}
%\newcommand{\Hbar}{\bar{H}}
%\newcommand{\Ibar}{\bar{I}}
%\newcommand{\Jbar}{\bar{J}}
%\newcommand{\Kbar}{\bar{K}}
%\newcommand{\Lbar}{\bar{L}}
%\newcommand{\Mbar}{\bar{M}}
%\newcommand{\Nbar}{\bar{N}}
%\newcommand{\Obar}{\bar{O}}
%\newcommand{\Pbar}{\bar{P}}
%\newcommand{\Qbar}{\bar{Q}}
%\newcommand{\Rbar}{\bar{R}}
%\newcommand{\Sbar}{\bar{S}}
%\newcommand{\Tbar}{\bar{T}}
%\newcommand{\Ubar}{\bar{U}}
%\newcommand{\Vbar}{\bar{V}}
%\newcommand{\Wbar}{\bar{W}}
%\newcommand{\Xbar}{\bar{X}}
%\newcommand{\Ybar}{\bar{Y}}
%\newcommand{\Zbar}{\bar{Z}}
%
%\newcommand{\Abbar}{\bar{\Ab}}
%\newcommand{\Bbbar}{\bar{\Bb}}
%\newcommand{\Cbbar}{\bar{\Cb}}
%\newcommand{\Dbbar}{\bar{\Db}}
%\newcommand{\Ebbar}{\bar{\Eb}}
%\newcommand{\Fbbar}{\bar{\Fb}}
%\newcommand{\Gbbar}{\bar{\Gb}}
%\newcommand{\Hbbar}{\bar{\Hb}}
%\newcommand{\Ibbar}{\bar{\Ib}}
%\newcommand{\Jbbar}{\bar{\Jb}}
%\newcommand{\Kbbar}{\bar{\Kb}}
%\newcommand{\Lbbar}{\bar{\Lb}}
%\newcommand{\Mbbar}{\bar{\Mb}}
%\newcommand{\Nbbar}{\bar{\Nb}}
%\newcommand{\Obbar}{\bar{\Ob}}
%\newcommand{\Pbbar}{\bar{\Pb}}
%\newcommand{\Qbbar}{\bar{\Qb}}
%\newcommand{\Rbbar}{\bar{\Rb}}
%\newcommand{\Sbbar}{\bar{\Sb}}
%\newcommand{\Tbbar}{\bar{\Tb}}
%\newcommand{\Ubbar}{\bar{\Ub}}
%\newcommand{\Vbbar}{\bar{\Vb}}
%\newcommand{\Wbbar}{\bar{\Wb}}
%\newcommand{\Xbbar}{\bar{\Xb}}
%\newcommand{\Ybbar}{\bar{\Yb}}
%\newcommand{\Zbbar}{\bar{\Zb}}
%
%\newcommand{\A}{\hat{A}}
%\newcommand{\Bhat}{\hat{B}}
%\newcommand{\Chat}{\hat{C}}
%\newcommand{\Dhat}{\hat{D}}
%\newcommand{\Ehat}{\hat{E}}
%\newcommand{\Fhat}{\hat{F}}
%\newcommand{\Ghat}{\hat{G}}
%\newcommand{\Hhat}{\hat{H}}
%\newcommand{\Ihat}{\hat{I}}
%\newcommand{\Jhat}{\hat{J}}
%\newcommand{\Khat}{\hat{K}}
%\newcommand{\Lhat}{\hat{L}}
%\newcommand{\Mhat}{\hat{M}}
%\newcommand{\Nhat}{\hat{N}}
%\newcommand{\Ohat}{\hat{O}}
%\newcommand{\Phat}{\hat{P}}
%\newcommand{\Qhat}{\hat{Q}}
%\newcommand{\Rhat}{\hat{R}}
%\newcommand{\Shat}{\hat{S}}
%\newcommand{\That}{\hat{T}}
%\newcommand{\Uhat}{\hat{U}}
%\newcommand{\Vhat}{\hat{V}}
%\newcommand{\What}{\hat{W}}
%\newcommand{\Xhat}{\hat{X}}
%\newcommand{\Yhat}{\hat{Y}}
%\newcommand{\Zhat}{\hat{Z}}
%
%\newcommand{\Abhat}{\hat{\Ab}}
%\newcommand{\Bbhat}{\hat{\Bb}}
%\newcommand{\Cbhat}{\hat{\Cb}}
%\newcommand{\Dbhat}{\hat{\Db}}
%\newcommand{\Ebhat}{\hat{\Eb}}
%\newcommand{\Fbhat}{\hat{\Fb}}
%\newcommand{\Gbhat}{\hat{\Gb}}
%\newcommand{\Hbhat}{\hat{\Hb}}
%\newcommand{\Ibhat}{\hat{\Ib}}
%\newcommand{\Jbhat}{\hat{\Jb}}
%\newcommand{\Kbhat}{\hat{\Kb}}
%\newcommand{\Lbhat}{\hat{\Lb}}
%\newcommand{\Mbhat}{\hat{\Mb}}
%\newcommand{\Nbhat}{\hat{\Nb}}
%\newcommand{\Obhat}{\hat{\Ob}}
%\newcommand{\Pbhat}{\hat{\Pb}}
%\newcommand{\Qbhat}{\hat{\Qb}}
%\newcommand{\Rbhat}{\hat{\Rb}}
%\newcommand{\Sbhat}{\hat{\Sb}}
%\newcommand{\Tbhat}{\hat{\Tb}}
%\newcommand{\Ubhat}{\hat{\Ub}}
%\newcommand{\Vbhat}{\hat{\Vb}}
%\newcommand{\Wbhat}{\hat{\Wb}}
%\newcommand{\Xbhat}{\hat{\Xb}}
%\newcommand{\Ybhat}{\hat{\Yb}}
%\newcommand{\Zbhat}{\hat{\Zb}}
%
%\newcommand{\Acal}{\mathcal{A}}
%\newcommand{\Bcal}{\mathcal{B}}
%\newcommand{\Ccal}{\mathcal{C}}
%\newcommand{\Dcal}{\mathcal{D}}
%\newcommand{\Ecal}{\mathcal{E}}
%\newcommand{\Fcal}{\mathcal{F}}
%\newcommand{\Gcal}{\mathcal{G}}
%\newcommand{\Hcal}{\mathcal{H}}
%\newcommand{\Ical}{\mathcal{I}}
%\newcommand{\Jcal}{\mathcal{J}}
%\newcommand{\Kcal}{\mathcal{K}}
%\newcommand{\Lcal}{\mathcal{L}}
%\newcommand{\Mcal}{\mathcal{M}}
%\newcommand{\Ncal}{\mathcal{N}}
%\newcommand{\Ocal}{\mathcal{O}}
%\newcommand{\Pcal}{\mathcal{P}}
%\newcommand{\Qcal}{\mathcal{Q}}
%\newcommand{\Rcal}{\mathcal{R}}
%\newcommand{\Scal}{{\mathcal{S}}}
%\newcommand{\Tcal}{{\mathcal{T}}}
%\newcommand{\Ucal}{\mathcal{U}}
%\newcommand{\Vcal}{\mathcal{V}}
%\newcommand{\Wcal}{\mathcal{W}}
%\newcommand{\Xcal}{\mathcal{X}}
%\newcommand{\Ycal}{\mathcal{Y}}
%\newcommand{\Zcal}{\mathcal{Z}}
%
%\newcommand{\Atilde}{\widetilde{A}}
%\newcommand{\Btilde}{\widetilde{B}}
%\newcommand{\Ctilde}{\widetilde{C}}
%\newcommand{\Dtilde}{\widetilde{D}}
%\newcommand{\Etilde}{\widetilde{E}}
%\newcommand{\Ftilde}{\widetilde{F}}
%\newcommand{\Gtilde}{\widetilde{G}}
%\newcommand{\Htilde}{\widetilde{H}}
%\newcommand{\Itilde}{\widetilde{I}}
%\newcommand{\Jtilde}{\widetilde{J}}
%\newcommand{\Ktilde}{\widetilde{K}}
%\newcommand{\Ltilde}{\widetilde{L}}
%\newcommand{\Mtilde}{\widetilde{M}}
%\newcommand{\Ntilde}{\widetilde{N}}
%\newcommand{\Otilde}{\widetilde{O}}
%\newcommand{\Ptilde}{\widetilde{P}}
%\newcommand{\Qtilde}{\widetilde{Q}}
%\newcommand{\Rtilde}{\widetilde{R}}
%\newcommand{\Stilde}{\widetilde{S}}
%\newcommand{\Ttilde}{\widetilde{T}}
%\newcommand{\Utilde}{\widetilde{U}}
%\newcommand{\Vtilde}{\widetilde{V}}
%\newcommand{\Wtilde}{\widetilde{W}}
%\newcommand{\Xtilde}{\widetilde{X}}
%\newcommand{\Ytilde}{\widetilde{Y}}
%\newcommand{\Ztilde}{\widetilde{Z}}
%
%
%%%%%%%%% Widely accepted definitions %%%%%%%%%%%%%%%
%
%\newcommand{\CC}{\mathbb{C}} % Complex numbers
%\newcommand{\EE}{\mathbb{E}} % Expectation
%\newcommand{\II}{\mathbb{I}} % Delta Indicator
%\newcommand{\VV}{\mathbb{V}} % Variance
%\newcommand{\KK}{\mathbb{K}} % Arbitrary field
%\newcommand{\MM}{\mathbb{M}} % Median
%\newcommand{\NN}{\mathbb{N}} % Natural numbers
%\newcommand{\PP}{\mathbb{P}} % Probability
%\newcommand{\QQ}{\mathbb{Q}} % Rationals
%\newcommand{\RR}{\mathbb{R}} % Real numbers
%\newcommand{\ZZ}{\mathbb{Z}} % Integers
%\newcommand{\XX}{\mathbb{X}} % Integers
%\newcommand{\YY}{\mathbb{Y}} % Integers
%
%\newcommand{\one}{\mathbf{1}}  % Identity
%\newcommand{\zero}{\mathbf{0}} % Zero
%%\newcommand{\TRUE}{\mathbf{TRUE}}  % True
%%\newcommand{\FALSE}{\mathbf{FALSE}}  % False
%
%\newcommand*{\mini}{\mathop{\mathrm{minimize}}}
%\newcommand*{\maxi}{\mathop{\mathrm{maximize}}}
%\newcommand*{\argmin}{\mathop{\mathrm{argmin}}}
%\newcommand*{\argmax}{\mathop{\mathrm{argmax}}}
%\newcommand{\sgn}{\mathop{\mathrm{sign}}}
%\newcommand{\tr}{\mathop{\mathrm{tr}}}
%\newcommand{\diag}{\mathop{\mathrm{diag}}}
%\newcommand{\rank}{\mathop{\mathrm{rank}}}
%%\newcommand{\ovec}{\mathop{\mathrm{vec}}}
%\newcommand{\traj}{\mathop{\mathrm{Traj}}}
%\newcommand*{\cov}{\mathrm{Cov}}
%\newcommand*{\conv}{\mathrm{conv}}
%\newcommand*{\const}{\mathrm{constant}}
%
%%%%%%%%% Bold Greek Letters %%%%%%%%%%%%%%%
%\newcommand{\sigmab}{\bm{\sigma}}
%\newcommand{\Sigmab}{\mathbf{\Sigma}}
%
%
%%%%%%%%% Mess around with LaTeX %%%%%%%%%%%%%%%
%
%%% Some style files might actually define these variables.
%%% So don't mess with them if they are already defined
%
%\ifx\BlackBox\undefined
%\newcommand{\BlackBox}{\rule{1.5ex}{1.5ex}}  % end of proof
%\fi
%
%\ifx\QED\undefined
%\def\QED{~\rule[-1pt]{5pt}{5pt}\par\medskip}
%\fi
%
%\ifx\proof\undefined
%\newenvironment{proof}{\par\noindent{\bf Proof\ }}{\hfill\BlackBox\\[2mm]}
%%\newenvironment{proof}{\emph{Proof. }}{ \hfill \QED}
%\fi
%
%\ifx\theorem\undefined
%\newtheorem{theorem}{Theorem}
%\newtheorem{example}{Example}
%\newtheorem{property}{Property}
%\newtheorem{lemma}[theorem]{Lemma}
%\newtheorem{proposition}[theorem]{Proposition}
%\newtheorem{remark}[theorem]{Remark}
%\newtheorem{corollary}[theorem]{Corollary}
%\newtheorem{definition}[theorem]{Definition}
%\newtheorem{conjecture}[theorem]{Conjecture}
%\fi
%
%\ifx\axiom\undefined
%\newtheorem{axiom}[theorem]{Axiom}
%\fi
%
%%%%%%%%% Utility functions %%%%%%%%%%%%%%%
%
%\newcommand{\eq}[1]{(\ref{#1})}
%\newcommand{\mymatrix}[2]{\left[\begin{array}{#1} #2 \end{array}\right]}
%\newcommand{\mychoose}[2]{\left(\begin{array}{c} #1 \\ #2 \end{array}\right)}
%\newcommand{\mydet}[1]{\det\left[ #1 \right]}
%\newcommand{\sembrack}[1]{[\![#1]\!]}
%
%\newcommand{\ea}{\emph{et al.}}
%\newcommand{\eg}{\emph{e.g.}}
%\newcommand{\ie}{\emph{i.e.}}
%\newcommand{\iid}{\emph{iid}}
%
%%\newcommand{\alex}[1]{{\bf ALEX: \uppercase{#1}}}
%%\newcommand{\vishy}[1]{{\bf VISHY: \uppercase{#1}}}
%%\newcommand{\rene}[1]{{\bf RENE: \uppercase{#1}}}
%%\newcommand{\karsten}[1]{{\bf KARSTEN: \uppercase{#1}}}
%
%%%%%%%%% Specific symbols for this project %%%%%%%%%%%%%%%
%
%\newcommand{\methodname}{KDE}
%\newcommand{\ind}{\boldsymbol{\mathsf{I}}}
%
%\newcommand{\hsic}{\mathrm{HSIC}}
%\newcommand{\mmd}{\mathrm{MMD}}
%
%%\newcommand{\tDiag}{\textsf{Diag}}
%%\newcommand{\tTr}{\textsf{Tr}}
%%\newcommand{\tE}{\textsf{E}}
%%\newcommand{\tVec}{\textsf{Vec}}
%%\newcommand{\tRank}{\textsf{Rank}}
%
%%%%%%%%%%%%%%%%%%%%%%%%%%%%%%%%%%%%%%%%%%%%%%%%%%%%%%%%%%%%%%%%%%%%%%%%%%%%
%% math symbols and commands
%%%%%%%%%%%%%%%%%%%%%%%%%%%%%%%%%%%%%%%%%%%%%%%%%%%%%%%%%%%%%%%%%%%%%%%%%%%%
%%\newcommand{\eq}[1]{(\ref{#1})}
%%\newcommand{\mymatrix}[2]{\left[\begin{array}{#1} #2 \end{array}\right]}
%
%%brackets
%\newcommand{\inner}[2]{\left\langle #1,#2 \right\rangle}
%\newcommand{\rbr}[1]{\left(#1\right)}
%\newcommand{\sbr}[1]{\left[#1\right]}
%\newcommand{\cbr}[1]{\left\{#1\right\}}
%\newcommand{\nbr}[1]{\left\|#1\right\|}
%\newcommand{\abr}[1]{\left|#1\right|}
%\newcommand{\smallfrac}[2]{{\textstyle \frac{#1}{#2}}}
%\renewcommand{\url}[1]{{\sffamily #1}}
%\newcommand{\arow}[2]{#1_{#2\cdot}}
%\newcommand{\acol}[2]{#1_{\cdot#2}}
%
%\newcommand{\twoco}[1]{\multicolumn{2}{c|}{#1}}
%\def\ci{\perp\!\!\!\perp}
%
%
%%%% Local Variables:
%%%% mode: latex
%%%% TeX-master: "./SIGMOD06/SIGMODB06.tex"
%%%% End:
%
%
%% by xinhua
%\newcommand{\delhat}{\hat{\delta}}
%\newcommand{\delbar}{\bar{\delta}}
%\newcommand{\HH}{\mathbb{H}} % Arbitrary field
%\newcommand{\ve}{\varepsilon}
%\newcommand{\dom}{\mathrm{dom}}
%
%%%% Local Variables:
%%%% mode: latex
%%%% TeX-master: "./SIGMOD06/SIGMODB06.tex"
%%%% End:
%%%%%%%%%%%%%%%%%%%%%%%%%%%%%%%%%%%%%%%%%%%%%%%%%%%%%%%%%%%%%%%%%%%%%%%%%%%
%%                               PROOF, THEOREM, and FRIENDS
%%%%%%%%%%%%%%%%%%%%%%%%%%%%%%%%%%%%%%%%%%%%%%%%%%%%%%%%%%%%%%%%%%%%%%%%%%%%
%
%%\newcommand{\BlackBox}{\rule{1.5ex}{1.5ex}}  % end of proof
%%\newenvironment{proof}{\par\noindent{\bf Proof.\ }}{\hfill\BlackBox\\[2mm]}
%
%%\newcounter{all_counter}
%%\newcounter{lemma_counter}
%%\newcounter{proposition_counter}
%%\newcounter{remark_counter}
%%\newcounter{corollary_counter}
%%\newcounter{definition_counter}
%%\newcounter{conjecture_counter}
%%\newcounter{axiom_counter}
%%
%%\setcounter{lemma_counter}{0} \setcounter{proposition_counter}{0}
%%\setcounter{remark_counter}{0} \setcounter{corollary_counter}{0}
%%\setcounter{definition_counter}{0}
%%\setcounter{conjecture_counter}{0} \setcounter{axiom_counter}{0}
%
%%\newtheorem{lemma}[lemma_counter]{Lemma}
%%\newtheorem{proposition}[proposition_counter]{Proposition}
%%\newtheorem{remark}[remark_counter]{Remark}
%%\newtheorem{corollary}[corollary_counter]{Corollary}
%%\newtheorem{definition}[definition_counter]{Definition}
%%\newtheorem{conjecture}[conjecture_counter]{Conjecture}
%%\newtheorem{axiom}[axiom_counter]{Axiom}
%
%\newtheorem{assumption}[theorem]{Assumption}
%
%\newcommand{\ahat}{\hat{a}}
%\newcommand{\bhat}{\hat{b}}
%\newcommand{\chat}{\hat{c}}
%\newcommand{\dhat}{\hat{d}}
%\newcommand{\ehat}{\hat{e}}
%\newcommand{\fhat}{\hat{f}}
%\newcommand{\ghat}{\hat{g}}
%\newcommand{\hhat}{\hat{h}}
%\newcommand{\ihat}{\hat{i}}
%\newcommand{\jhat}{\hat{j}}
%\newcommand{\khat}{\hat{k}}
%\newcommand{\lhat}{\hat{l}}
%\newcommand{\mhat}{\hat{m}}
%\newcommand{\nhat}{\hat{n}}
%\newcommand{\ohat}{\hat{o}}
%\newcommand{\phat}{\hat{p}}
%\newcommand{\qhat}{\hat{q}}
%\newcommand{\rhat}{\hat{r}}
%\newcommand{\shat}{\hat{s}}
%\newcommand{\that}{\hat{t}}
%\newcommand{\uhat}{\hat{u}}
%\newcommand{\vhat}{\hat{v}}
%\newcommand{\what}{\hat{w}}
%\newcommand{\xhat}{\hat{x}}
%\newcommand{\yhat}{\hat{y}}
%\newcommand{\zhat}{\hat{z}}
%
%\newcommand{\abhat}{\hat{\ab}}
%\newcommand{\bbhat}{\hat{\bb}}
%\newcommand{\cbhat}{\hat{\cb}}
%\newcommand{\dbhat}{\hat{\db}}
%\newcommand{\ebhat}{\hat{\eb}}
%\newcommand{\fbhat}{\hat{\fb}}
%\newcommand{\gbhat}{\hat{\gb}}
%\newcommand{\hbhat}{\hat{\hb}}
%\newcommand{\ibhat}{\hat{\ib}}
%\newcommand{\jbhat}{\hat{\jb}}
%\newcommand{\kbhat}{\hat{\kb}}
%\newcommand{\lbhat}{\hat{\lb}}
%\newcommand{\mbhat}{\hat{\mb}}
%\newcommand{\nbhat}{\hat{\nb}}
%\newcommand{\obhat}{\hat{\ob}}
%\newcommand{\pbhat}{\hat{\pb}}
%\newcommand{\qbhat}{\hat{\qb}}
%\newcommand{\rbhat}{\hat{\rb}}
%\newcommand{\sbhat}{\hat{\sb}}
%\newcommand{\tbhat}{\hat{\tb}}
%\newcommand{\ubhat}{\hat{\ub}}
%\newcommand{\vbhat}{\hat{\vb}}
%\newcommand{\wbhat}{\hat{\wb}}
%\newcommand{\xbhat}{\hat{\xb}}
%\newcommand{\ybhat}{\hat{\yb}}
%\newcommand{\zbhat}{\hat{\zb}}
%
%\renewcommand{\vec}[1]{\mathbf{\boldsymbol{#1}}}
%
%\newcommand{\avec}{\vec{a}}
%\newcommand{\bvec}{\vec{b}}
%\newcommand{\cvec}{\vec{c}}
%\newcommand{\dvec}{\vec{d}}
%\newcommand{\evec}{\vec{e}}
%\newcommand{\fvec}{\vec{f}}
%\newcommand{\gvec}{\vec{g}}
%\newcommand{\hvec}{\vec{h}}
%\newcommand{\ivec}{\vec{i}}
%\newcommand{\jvec}{\vec{j}}
%\newcommand{\kvec}{\vec{k}}
%\newcommand{\lvec}{\vec{l}}
%\newcommand{\mvec}{\vec{m}}
%\newcommand{\nvec}{\vec{n}}
%\newcommand{\ovec}{\vec{o}}
%\newcommand{\pvec}{\vec{p}}
%\newcommand{\qvec}{\vec{q}}
%\newcommand{\rvec}{\vec{r}}
%\newcommand{\svec}{\vec{s}}
%\newcommand{\tvec}{\vec{t}}
%\newcommand{\uvec}{\vec{u}}
%\newcommand{\vvec}{\vec{v}}
%\newcommand{\wvec}{\vec{w}}
%\newcommand{\xvec}{\vec{x}}
%\newcommand{\yvec}{\vec{y}}
%\newcommand{\zvec}{\vec{z}}
%
%\newcommand{\Avec}{\vec{A}}
%\newcommand{\Bvec}{\vec{B}}
%\newcommand{\Cvec}{\vec{C}}
%\newcommand{\Dvec}{\vec{D}}
%\newcommand{\Evec}{\vec{E}}
%\newcommand{\Fvec}{\vec{F}}
%\newcommand{\Gvec}{\vec{G}}
%\newcommand{\Hvec}{\vec{H}}
%\newcommand{\Ivec}{\vec{I}}
%\newcommand{\Jvec}{\vec{J}}
%\newcommand{\Kvec}{\vec{K}}
%\newcommand{\Lvec}{\vec{L}}
%\newcommand{\Mvec}{\vec{M}}
%\newcommand{\Nvec}{\vec{N}}
%\newcommand{\Ovec}{\vec{O}}
%\newcommand{\Pvec}{\vec{P}}
%\newcommand{\Qvec}{\vec{Q}}
%\newcommand{\Rvec}{\vec{R}}
%\newcommand{\Svec}{\vec{S}}
%\newcommand{\Tvec}{\vec{T}}
%\newcommand{\Uvec}{\vec{U}}
%\newcommand{\Vvec}{\vec{V}}
%\newcommand{\Wvec}{\vec{W}}
%\newcommand{\Xvec}{\vec{X}}
%\newcommand{\Yvec}{\vec{Y}}
%\newcommand{\Zvec}{\vec{Z}}
%
%
%\newcommand{\alphavec}{\vec{\alpha}}
%\newcommand{\betavec}{\vec{\beta}}
%\newcommand{\deltavec}{\vec{\delta}}
%\newcommand{\etavec}{\vec{\eta}}
%\newcommand{\phivec}{\vec{\phi}}
%\newcommand{\psivec}{\vec{\psi}}
%\newcommand{\thetavec}{\vec{\theta}}
%\newcommand{\muvec}{\vec{\mu}}
%\newcommand{\xivec}{\vec{\xi}}
%
%
%\newcommand{\alphab}{\boldsymbol{\alpha}}
%\newcommand{\betab}{\boldsymbol{\beta}}
%\newcommand{\gammab}{\boldsymbol{\gamma}}
%\newcommand{\thetab}{\boldsymbol{\theta}}
%\newcommand{\mub}{\boldsymbol{\mu}}
%\newcommand{\xib}{\boldsymbol{\xi}}
%\newcommand{\Deltab}{\boldsymbol{\Delta}}
%\newcommand{\Pib}{\boldsymbol{\Pi}}
%\newcommand{\etab}{\boldsymbol{\eta}}
%\newcommand{\taub}{\boldsymbol{\tau}}
%\newcommand{\lambdab}{\boldsymbol{\lambda}}
%\newcommand{\pitil}{\tilde{\pi}}
%
%\newcommand{\rmd}{\mathrm{d}}
%\newcommand{\HSIC}{\mathrm{HSIC}}
%
%\newcommand{\dir}{\mathrm{Dir}}
%\newcommand{\dirp}{\mathrm{DP}}
%\newcommand{\multi}{\mathrm{Multi}}
%\newcommand{\gam}{\mathrm{Gamma}}
%\newcommand{\var}{\mathrm{Var}}
%\newcommand{\cor}{\mathrm{Corr}}
%\newcommand{\bet}{\mathrm{Beta}}
%
%\newcommand{\ra}{\rightarrow}
%
%\newcommand{\idf}{{\tt idf}}
%
%\newcommand{\tp}{\text{tp}}
%\newcommand{\fn}{\text{fn}}
%\newcommand{\fp}{\text{fp}}
%\newcommand{\tpb}{\overline{\text{tp}}}
%\newcommand{\fnb}{\overline{\text{fn}}}
%\newcommand{\fpb}{\overline{\text{fp}}}
%\newcommand{\pre}{\text{Pre}}
%\newcommand{\rec}{\text{Rec}}
%\newcommand{\fs}{\text{F-score}}
%\newcommand{\opt}{\text{opt}}
%
%\newcommand{\topics}{{\tt topics}}
%\newcommand{\industries}{{\tt industries}}
%\newcommand{\regions}{{\tt regions}}
%\newcommand{\ip}{{\tt ip}}
%\newcommand{\category}{{\tt category}}
%
%\newcommand{\grad}{{\nabla}}
%\newcommand{\gradbar}{{\bar{\nabla}}}
%\newcommand{\mbarr}{{\overline{M(r)}}}
%\newcommand{\card}{{\mathrm{card }}}
%\newcommand{\MED}{{\text{MED}}}
%\newcommand{\mubar}{\bar{\mu}}
%\newcommand{\emp}{{\mathrm{emp}}}
%\newcommand{\core}{{\mathrm{core}}}
%\newcommand{\zerob}{{\mathbf{0}}}
%\newcommand{\co}{{\mathrm{co}}}
%\newcommand{\argsup}{\mathop{\mathrm{argsup}}}
%\newcommand{\intr}{{\mathrm{int}}}
%\newcommand{\bd}{{\mathrm{bd}}}
%\newcommand{\BMRM}{{BMRM}}
%\newcommand{\where}{{\quad \text{where} \quad}}
%\newcommand{\lcg}{{\textit{l.c.g}}}
%\newcommand{\gradtil}{{\tilde{\grad}}}
%\newcommand{\alphabtil}{{\tilde{\alphab}}}
%\newcommand{\gstar}{{g^{\star}}}
%\newcommand{\fstar}{{f^{\star}}}
%
%
%\newcommand{\sqrttwo}{{\sqrt{2}}}
%\newcommand{\invsqrttwo}{{\frac{1}{\sqrt{2}}}}
%\newcommand{\invtwo}{{\frac{1}{2}}}
%\newcommand{\phihat}{{\hat{\phi}}}
%\newcommand{\cp}{\mathrm{cp}}
%\newcommand{\liblinear}{{\sf liblinear}}
%\newcommand{\lsbmrm}{{\sf ls-bmrm}}
%\newcommand{\qpbmrm}{{\sf qp-bmrm}}
%\newcommand{\pegasos}{{\sf pegasos}}
%\newcommand{\pegan}{{\sf pegasos-$n$}}
%\newcommand{\pegaone}{{\sf pegasos-1}}
%\newcommand{\nest}{{\sf Pragam}}
%\newcommand{\nestb}{{\sf Pragam-b}}
%\newcommand{\err}{{\sf err}}



\newcommand{\ab}{\mathbf{a}}
\newcommand{\bb}{\mathbf{b}}
\newcommand{\cbb}{\mathbf{c}}
\newcommand{\db}{\mathbf{d}}
\newcommand{\eb}{\mathbf{e}}
\newcommand{\fb}{\mathbf{f}}
\newcommand{\gb}{\mathbf{g}}
\newcommand{\hb}{\mathbf{h}}
\newcommand{\ib}{\mathbf{i}}
\newcommand{\jb}{\mathbf{j}}
\newcommand{\kb}{\mathbf{k}}
\newcommand{\lb}{\mathbf{l}}
\newcommand{\mb}{\mathbf{m}}
\newcommand{\nbb}{\mathbf{n}}
\newcommand{\ob}{\mathbf{o}}
\newcommand{\pb}{\mathbf{p}}
\newcommand{\qb}{\mathbf{q}}
\newcommand{\rb}{\mathbf{r}}
\newcommand{\sbb}{\mathbf{s}}
\newcommand{\tb}{\mathbf{t}}
\newcommand{\ub}{\mathbf{u}}
\newcommand{\vb}{\mathbf{v}}
\newcommand{\wb}{\mathbf{w}}
\newcommand{\xb}{\mathbf{x}}
\newcommand{\yb}{\mathbf{y}}
\newcommand{\zb}{\mathbf{z}}

\newcommand{\abtil}{\tilde{\ab}}
\newcommand{\bbtil}{\tilde{\bb}}
\newcommand{\cbtil}{\tilde{\cbb}}
\newcommand{\dbtil}{\tilde{\db}}
\newcommand{\ebtil}{\tilde{\eb}}
\newcommand{\fbtil}{\tilde{\fb}}
\newcommand{\gbtil}{\tilde{\gb}}
\newcommand{\hbtil}{\tilde{\hb}}
\newcommand{\ibtil}{\tilde{\ib}}
\newcommand{\jbtil}{\tilde{\jb}}
\newcommand{\kbtil}{\tilde{\kb}}
\newcommand{\lbtil}{\tilde{\lb}}
\newcommand{\mbtil}{\tilde{\mb}}
\newcommand{\nbtil}{\tilde{\nbb}}
\newcommand{\obtil}{\tilde{\ob}}
\newcommand{\pbtil}{\tilde{\pb}}
\newcommand{\qbtil}{\tilde{\qb}}
\newcommand{\rbtil}{\tilde{\rb}}
\newcommand{\sbtil}{\tilde{\sbb}}
\newcommand{\tbtil}{\tilde{\tb}}
\newcommand{\ubtil}{\tilde{\ub}}
\newcommand{\vbtil}{\tilde{\vb}}
\newcommand{\wbtil}{\tilde{\wb}}
\newcommand{\xbtil}{\tilde{\xb}}
\newcommand{\ybtil}{\tilde{\yb}}
\newcommand{\zbtil}{\tilde{\zb}}


\newcommand{\atil}{\tilde{a}}
\newcommand{\btil}{\tilde{b}}
\newcommand{\ctil}{\tilde{c}}
\newcommand{\dtil}{\tilde{d}}
\newcommand{\etil}{\tilde{e}}
\newcommand{\ftil}{\tilde{f}}
\newcommand{\gtil}{\tilde{g}}
\newcommand{\htil}{\tilde{h}}
\newcommand{\itil}{\tilde{i}}
\newcommand{\jtil}{\tilde{j}}
\newcommand{\ktil}{\tilde{k}}
\newcommand{\ltil}{\tilde{l}}
\newcommand{\mtil}{\tilde{m}}
\newcommand{\ntil}{\tilde{n}}
\newcommand{\otil}{\tilde{o}}
\newcommand{\ptil}{\tilde{p}}
\newcommand{\qtil}{\tilde{q}}
\newcommand{\rtil}{\tilde{r}}
\newcommand{\stil}{\tilde{s}}
\newcommand{\ttil}{\tilde{t}}
\newcommand{\util}{\tilde{u}}
\newcommand{\vtil}{\tilde{v}}
\newcommand{\wtil}{\tilde{w}}
\newcommand{\xtil}{\tilde{x}}
\newcommand{\ytil}{\tilde{y}}
\newcommand{\ztil}{\tilde{z}}

\newcommand{\Atil}{\tilde{A}}
\newcommand{\Btil}{\tilde{B}}
\newcommand{\Ctil}{\tilde{C}}
\newcommand{\Dtil}{\tilde{D}}
\newcommand{\Etil}{\tilde{E}}
\newcommand{\Ftil}{\tilde{F}}
\newcommand{\Gtil}{\tilde{G}}
\newcommand{\Htil}{\tilde{H}}
\newcommand{\Itil}{\tilde{I}}
\newcommand{\Jtil}{\tilde{J}}
\newcommand{\Ktil}{\tilde{K}}
\newcommand{\Ltil}{\tilde{L}}
\newcommand{\Mtil}{\tilde{M}}
\newcommand{\Ntil}{\tilde{N}}
\newcommand{\Otil}{\tilde{O}}
\newcommand{\Ptil}{\tilde{P}}
\newcommand{\Qtil}{\tilde{Q}}
\newcommand{\Rtil}{\tilde{R}}
\newcommand{\Stil}{\tilde{S}}
\newcommand{\Ttil}{\tilde{T}}
\newcommand{\Util}{\tilde{U}}
\newcommand{\Vtil}{\tilde{V}}
\newcommand{\Wtil}{\tilde{W}}
\newcommand{\Xtil}{\tilde{X}}
\newcommand{\Ytil}{\tilde{Y}}
\newcommand{\Ztil}{\tilde{Z}}

\newcommand{\abar}{\bar{a}}
\newcommand{\bbar}{\bar{b}}
\newcommand{\cbar}{\bar{c}}
\newcommand{\dbar}{\bar{d}}
\newcommand{\ebar}{\bar{e}}
\newcommand{\fbar}{\bar{f}}
\newcommand{\gbar}{\bar{g}}
\newcommand{\hbr}{\bar{h}}
\newcommand{\ibar}{\bar{i}}
\newcommand{\jbar}{\bar{j}}
\newcommand{\kbar}{\bar{k}}
\newcommand{\lbar}{\bar{l}}
\newcommand{\mbar}{\bar{m}}
\newcommand{\nbar}{\bar{n}}
\newcommand{\oobar}{\bar{o}}
\newcommand{\pbar}{\bar{p}}
\newcommand{\qbar}{\bar{q}}
\newcommand{\rbar}{\bar{r}}
\newcommand{\sbar}{\bar{s}}
\newcommand{\tbar}{\bar{t}}
\newcommand{\ubar}{\bar{u}}
\newcommand{\vbar}{\bar{v}}
\newcommand{\wbar}{\bar{w}}
\newcommand{\xbar}{\bar{x}}
\newcommand{\ybar}{\bar{y}}
\newcommand{\zbar}{\bar{z}}

\newcommand{\abbar}{\bar{\ab}}
\newcommand{\bbbar}{\bar{\bb}}
\newcommand{\cbbar}{\bar{\cb}}
\newcommand{\dbbar}{\bar{\db}}
\newcommand{\ebbar}{\bar{\eb}}
\newcommand{\fbbar}{\bar{\fb}}
\newcommand{\gbbar}{\bar{\gb}}
\newcommand{\hbbar}{\bar{\hb}}
\newcommand{\ibbar}{\bar{\ib}}
\newcommand{\jbbar}{\bar{\jb}}
\newcommand{\kbbar}{\bar{\kb}}
\newcommand{\lbbar}{\bar{\lb}}
\newcommand{\mbbar}{\bar{\mb}}
\newcommand{\nbbar}{\bar{\nbb}}
\newcommand{\obbar}{\bar{\ob}}
\newcommand{\pbbar}{\bar{\pb}}
\newcommand{\qbbar}{\bar{\qb}}
\newcommand{\rbbar}{\bar{\rb}}
\newcommand{\sbbar}{\bar{\sbb}}
\newcommand{\tbbar}{\bar{\tb}}
\newcommand{\ubbar}{\bar{\ub}}
\newcommand{\vbbar}{\bar{\vb}}
\newcommand{\wbbar}{\bar{\wb}}
\newcommand{\xbbar}{\bar{\xb}}
\newcommand{\ybbar}{\bar{\yb}}
\newcommand{\zbbar}{\bar{\zb}}

\newcommand{\Ab}{\mathbf{A}}
\newcommand{\Bb}{\mathbf{B}}
\newcommand{\Cb}{\mathbf{C}}
\newcommand{\Db}{\mathbf{D}}
\newcommand{\Eb}{\mathbf{E}}
\newcommand{\Fb}{\mathbf{F}}
\newcommand{\Gb}{\mathbf{G}}
\newcommand{\Hb}{\mathbf{H}}
\newcommand{\Ib}{\mathbf{I}}
\newcommand{\Jb}{\mathbf{J}}
\newcommand{\Kb}{\mathbf{K}}
\newcommand{\Lb}{\mathbf{L}}
\newcommand{\Mb}{\mathbf{M}}
\newcommand{\Nb}{\mathbf{N}}
\newcommand{\Ob}{\mathbf{O}}
\newcommand{\Pb}{\mathbf{P}}
\newcommand{\Qb}{\mathbf{Q}}
\newcommand{\Rb}{\mathbf{R}}
\newcommand{\Sbb}{\mathbf{S}}
\newcommand{\Tb}{\mathbf{T}}
\newcommand{\Ub}{\mathbf{U}}
\newcommand{\Vb}{\mathbf{V}}
\newcommand{\Wb}{\mathbf{W}}
\newcommand{\Xb}{\mathbf{X}}
\newcommand{\Yb}{\mathbf{Y}}
\newcommand{\Zb}{\mathbf{Z}}

\newcommand{\Abtil}{\tilde{\Ab}}
\newcommand{\Bbtil}{\tilde{\Bb}}
\newcommand{\Cbtil}{\tilde{\Cb}}
\newcommand{\Dbtil}{\tilde{\Db}}
\newcommand{\Ebtil}{\tilde{\Eb}}
\newcommand{\Fbtil}{\tilde{\Fb}}
\newcommand{\Gbtil}{\tilde{\Gb}}
\newcommand{\Hbtil}{\tilde{\Hb}}
\newcommand{\Ibtil}{\tilde{\Ib}}
\newcommand{\Jbtil}{\tilde{\Jb}}
\newcommand{\Kbtil}{\tilde{\Kb}}
\newcommand{\Lbtil}{\tilde{\Lb}}
\newcommand{\Mbtil}{\tilde{\Mb}}
\newcommand{\Nbtil}{\tilde{\Nb}}
\newcommand{\Obtil}{\tilde{\Ob}}
\newcommand{\Pbtil}{\tilde{\Pb}}
\newcommand{\Qbtil}{\tilde{\Qb}}
\newcommand{\Rbtil}{\tilde{\Rb}}
\newcommand{\Sbtil}{\tilde{\Sbb}}
\newcommand{\Tbtil}{\tilde{\Tb}}
\newcommand{\Ubtil}{\tilde{\Ub}}
\newcommand{\Vbtil}{\tilde{\Vb}}
\newcommand{\Wbtil}{\tilde{\Wb}}
\newcommand{\Xbtil}{\tilde{\Xb}}
\newcommand{\Ybtil}{\tilde{\Yb}}
\newcommand{\Zbtil}{\tilde{\Zb}}

\newcommand{\Abar}{\bar{A}}
\newcommand{\Bbar}{\bar{B}}
\newcommand{\Cbar}{\bar{C}}
\newcommand{\Dbar}{\bar{D}}
\newcommand{\Ebar}{\bar{E}}
\newcommand{\Fbar}{\bar{F}}
\newcommand{\Gbar}{\bar{G}}
\newcommand{\Hbar}{\bar{H}}
\newcommand{\Ibar}{\bar{I}}
\newcommand{\Jbar}{\bar{J}}
\newcommand{\Kbar}{\bar{K}}
\newcommand{\Lbar}{\bar{L}}
\newcommand{\Mbar}{\bar{M}}
\newcommand{\Nbar}{\bar{N}}
\newcommand{\Obar}{\bar{O}}
\newcommand{\Pbar}{\bar{P}}
\newcommand{\Qbar}{\bar{Q}}
\newcommand{\Rbar}{\bar{R}}
\newcommand{\Sbar}{\bar{S}}
\newcommand{\Tbar}{\bar{T}}
\newcommand{\Ubar}{\bar{U}}
\newcommand{\Vbar}{\bar{V}}
\newcommand{\Wbar}{\bar{W}}
\newcommand{\Xbar}{\bar{X}}
\newcommand{\Ybar}{\bar{Y}}
\newcommand{\Zbar}{\bar{Z}}

\newcommand{\Abbar}{\bar{\Ab}}
\newcommand{\Bbbar}{\bar{\Bb}}
\newcommand{\Cbbar}{\bar{\Cb}}
\newcommand{\Dbbar}{\bar{\Db}}
\newcommand{\Ebbar}{\bar{\Eb}}
\newcommand{\Fbbar}{\bar{\Fb}}
\newcommand{\Gbbar}{\bar{\Gb}}
\newcommand{\Hbbar}{\bar{\Hb}}
\newcommand{\Ibbar}{\bar{\Ib}}
\newcommand{\Jbbar}{\bar{\Jb}}
\newcommand{\Kbbar}{\bar{\Kb}}
\newcommand{\Lbbar}{\bar{\Lb}}
\newcommand{\Mbbar}{\bar{\Mb}}
\newcommand{\Nbbar}{\bar{\Nb}}
\newcommand{\Obbar}{\bar{\Ob}}
\newcommand{\Pbbar}{\bar{\Pb}}
\newcommand{\Qbbar}{\bar{\Qb}}
\newcommand{\Rbbar}{\bar{\Rb}}
\newcommand{\Sbbar}{\bar{\Sb}}
\newcommand{\Tbbar}{\bar{\Tb}}
\newcommand{\Ubbar}{\bar{\Ub}}
\newcommand{\Vbbar}{\bar{\Vb}}
\newcommand{\Wbbar}{\bar{\Wb}}
\newcommand{\Xbbar}{\bar{\Xb}}
\newcommand{\Ybbar}{\bar{\Yb}}
\newcommand{\Zbbar}{\bar{\Zb}}

\newcommand{\Ahat}{\hat{A}}
\newcommand{\Bhat}{\hat{B}}
\newcommand{\Chat}{\hat{C}}
\newcommand{\Dhat}{\hat{D}}
\newcommand{\Ehat}{\hat{E}}
\newcommand{\Fhat}{\hat{F}}
\newcommand{\Ghat}{\hat{G}}
\newcommand{\Hhat}{\hat{H}}
\newcommand{\Ihat}{\hat{I}}
\newcommand{\Jhat}{\hat{J}}
\newcommand{\Khat}{\hat{K}}
\newcommand{\Lhat}{\hat{L}}
\newcommand{\Mhat}{\hat{M}}
\newcommand{\Nhat}{\hat{N}}
\newcommand{\Ohat}{\hat{O}}
\newcommand{\Phat}{\hat{P}}
\newcommand{\Qhat}{\hat{Q}}
\newcommand{\Rhat}{\hat{R}}
\newcommand{\Shat}{\hat{S}}
\newcommand{\That}{\hat{T}}
\newcommand{\Uhat}{\hat{U}}
\newcommand{\Vhat}{\hat{V}}
\newcommand{\What}{\hat{W}}
\newcommand{\Xhat}{\hat{X}}
\newcommand{\Yhat}{\hat{Y}}
\newcommand{\Zhat}{\hat{Z}}

\newcommand{\Abhat}{\hat{\Ab}}
\newcommand{\Bbhat}{\hat{\Bb}}
\newcommand{\Cbhat}{\hat{\Cb}}
\newcommand{\Dbhat}{\hat{\Db}}
\newcommand{\Ebhat}{\hat{\Eb}}
\newcommand{\Fbhat}{\hat{\Fb}}
\newcommand{\Gbhat}{\hat{\Gb}}
\newcommand{\Hbhat}{\hat{\Hb}}
\newcommand{\Ibhat}{\hat{\Ib}}
\newcommand{\Jbhat}{\hat{\Jb}}
\newcommand{\Kbhat}{\hat{\Kb}}
\newcommand{\Lbhat}{\hat{\Lb}}
\newcommand{\Mbhat}{\hat{\Mb}}
\newcommand{\Nbhat}{\hat{\Nb}}
\newcommand{\Obhat}{\hat{\Ob}}
\newcommand{\Pbhat}{\hat{\Pb}}
\newcommand{\Qbhat}{\hat{\Qb}}
\newcommand{\Rbhat}{\hat{\Rb}}
\newcommand{\Sbhat}{\hat{\Sb}}
\newcommand{\Tbhat}{\hat{\Tb}}
\newcommand{\Ubhat}{\hat{\Ub}}
\newcommand{\Vbhat}{\hat{\Vb}}
\newcommand{\Wbhat}{\hat{\Wb}}
\newcommand{\Xbhat}{\hat{\Xb}}
\newcommand{\Ybhat}{\hat{\Yb}}
\newcommand{\Zbhat}{\hat{\Zb}}

\newcommand{\Acal}{\mathcal{A}}
\newcommand{\Bcal}{\mathcal{B}}
\newcommand{\Ccal}{\mathcal{C}}
\newcommand{\Dcal}{\mathcal{D}}
\newcommand{\Ecal}{\mathcal{E}}
\newcommand{\Fcal}{\mathcal{F}}
\newcommand{\Gcal}{\mathcal{G}}
\newcommand{\Hcal}{\mathcal{H}}
\newcommand{\Ical}{\mathcal{I}}
\newcommand{\Jcal}{\mathcal{J}}
\newcommand{\Kcal}{\mathcal{K}}
\newcommand{\Lcal}{\mathcal{L}}
\newcommand{\Mcal}{\mathcal{M}}
\newcommand{\Ncal}{\mathcal{N}}
\newcommand{\Ocal}{\mathcal{O}}
\newcommand{\Pcal}{\mathcal{P}}
\newcommand{\Qcal}{\mathcal{Q}}
\newcommand{\Rcal}{\mathcal{R}}
\newcommand{\Scal}{{\mathcal{S}}}
\newcommand{\Tcal}{{\mathcal{T}}}
\newcommand{\Ucal}{\mathcal{U}}
\newcommand{\Vcal}{\mathcal{V}}
\newcommand{\Wcal}{\mathcal{W}}
\newcommand{\Xcal}{\mathcal{X}}
\newcommand{\Ycal}{\mathcal{Y}}
\newcommand{\Zcal}{\mathcal{Z}}

\newcommand{\Atilde}{\widetilde{A}}
\newcommand{\Btilde}{\widetilde{B}}
\newcommand{\Ctilde}{\widetilde{C}}
\newcommand{\Dtilde}{\widetilde{D}}
\newcommand{\Etilde}{\widetilde{E}}
\newcommand{\Ftilde}{\widetilde{F}}
\newcommand{\Gtilde}{\widetilde{G}}
\newcommand{\Htilde}{\widetilde{H}}
\newcommand{\Itilde}{\widetilde{I}}
\newcommand{\Jtilde}{\widetilde{J}}
\newcommand{\Ktilde}{\widetilde{K}}
\newcommand{\Ltilde}{\widetilde{L}}
\newcommand{\Mtilde}{\widetilde{M}}
\newcommand{\Ntilde}{\widetilde{N}}
\newcommand{\Otilde}{\widetilde{O}}
\newcommand{\Ptilde}{\widetilde{P}}
\newcommand{\Qtilde}{\widetilde{Q}}
\newcommand{\Rtilde}{\widetilde{R}}
\newcommand{\Stilde}{\widetilde{S}}
\newcommand{\Ttilde}{\widetilde{T}}
\newcommand{\Utilde}{\widetilde{U}}
\newcommand{\Vtilde}{\widetilde{V}}
\newcommand{\Wtilde}{\widetilde{W}}
\newcommand{\Xtilde}{\widetilde{X}}
\newcommand{\Ytilde}{\widetilde{Y}}
\newcommand{\Ztilde}{\widetilde{Z}}

\newcommand{\ahat}{\hat{a}}
\newcommand{\bhat}{\hat{b}}
\newcommand{\chat}{\hat{c}}
\newcommand{\dhat}{\hat{d}}
\newcommand{\ehat}{\hat{e}}
\newcommand{\fhat}{\hat{f}}
\newcommand{\ghat}{\hat{g}}
\newcommand{\hhat}{\hat{h}}
\newcommand{\ihat}{\hat{i}}
\newcommand{\jhat}{\hat{j}}
\newcommand{\khat}{\hat{k}}
\newcommand{\lhat}{\hat{l}}
\newcommand{\mhat}{\hat{m}}
\newcommand{\nhat}{\hat{n}}
\newcommand{\ohat}{\hat{o}}
\newcommand{\phat}{\hat{p}}
\newcommand{\qhat}{\hat{q}}
\newcommand{\rhat}{\hat{r}}
\newcommand{\shat}{\hat{s}}
\newcommand{\that}{\hat{t}}
\newcommand{\uhat}{\hat{u}}
\newcommand{\vhat}{\hat{v}}
\newcommand{\what}{\hat{w}}
\newcommand{\xhat}{\hat{x}}
\newcommand{\yhat}{\hat{y}}
\newcommand{\zhat}{\hat{z}}

\newcommand{\abhat}{\hat{\ab}}
\newcommand{\bbhat}{\hat{\bb}}
\newcommand{\cbhat}{\hat{\cb}}
\newcommand{\dbhat}{\hat{\db}}
\newcommand{\ebhat}{\hat{\eb}}
\newcommand{\fbhat}{\hat{\fb}}
\newcommand{\gbhat}{\hat{\gb}}
\newcommand{\hbhat}{\hat{\hb}}
\newcommand{\ibhat}{\hat{\ib}}
\newcommand{\jbhat}{\hat{\jb}}
\newcommand{\kbhat}{\hat{\kb}}
\newcommand{\lbhat}{\hat{\lb}}
\newcommand{\mbhat}{\hat{\mb}}
\newcommand{\nbhat}{\hat{\nb}}
\newcommand{\obhat}{\hat{\ob}}
\newcommand{\pbhat}{\hat{\pb}}
\newcommand{\qbhat}{\hat{\qb}}
\newcommand{\rbhat}{\hat{\rb}}
\newcommand{\sbhat}{\hat{\sb}}
\newcommand{\tbhat}{\hat{\tb}}
\newcommand{\ubhat}{\hat{\ub}}
\newcommand{\vbhat}{\hat{\vb}}
\newcommand{\wbhat}{\hat{\wb}}
\newcommand{\xbhat}{\hat{\xb}}
\newcommand{\ybhat}{\hat{\yb}}
\newcommand{\zbhat}{\hat{\zb}}




\renewcommand{\vec}[1]{\mathbf{\boldsymbol{#1}}}

\newcommand{\avec}{\vec{a}}
\newcommand{\bvec}{\vec{b}}
\newcommand{\cvec}{\vec{c}}
\newcommand{\dvec}{\vec{d}}
\newcommand{\evec}{\vec{e}}
\newcommand{\fvec}{\vec{f}}
\newcommand{\gvec}{\vec{g}}
\newcommand{\hvec}{\vec{h}}
\newcommand{\ivec}{\vec{i}}
\newcommand{\jvec}{\vec{j}}
\newcommand{\kvec}{\vec{k}}
\newcommand{\lvec}{\vec{l}}
\newcommand{\mvec}{\vec{m}}
\newcommand{\nvec}{\vec{n}}
\newcommand{\ovec}{\vec{o}}
\newcommand{\pvec}{\vec{p}}
\newcommand{\qvec}{\vec{q}}
\newcommand{\rvec}{\vec{r}}
\newcommand{\svec}{\vec{s}}
\newcommand{\tvec}{\vec{t}}
\newcommand{\uvec}{\vec{u}}
\newcommand{\vvec}{\vec{v}}
\newcommand{\wvec}{\vec{w}}
\newcommand{\xvec}{\vec{x}}
\newcommand{\yvec}{\vec{y}}
\newcommand{\zvec}{\vec{z}}

\newcommand{\Avec}{\vec{A}}
\newcommand{\Bvec}{\vec{B}}
\newcommand{\Cvec}{\vec{C}}
\newcommand{\Dvec}{\vec{D}}
\newcommand{\Evec}{\vec{E}}
\newcommand{\Fvec}{\vec{F}}
\newcommand{\Gvec}{\vec{G}}
\newcommand{\Hvec}{\vec{H}}
\newcommand{\Ivec}{\vec{I}}
\newcommand{\Jvec}{\vec{J}}
\newcommand{\Kvec}{\vec{K}}
\newcommand{\Lvec}{\vec{L}}
\newcommand{\Mvec}{\vec{M}}
\newcommand{\Nvec}{\vec{N}}
\newcommand{\Ovec}{\vec{O}}
\newcommand{\Pvec}{\vec{P}}
\newcommand{\Qvec}{\vec{Q}}
\newcommand{\Rvec}{\vec{R}}
\newcommand{\Svec}{\vec{S}}
\newcommand{\Tvec}{\vec{T}}
\newcommand{\Uvec}{\vec{U}}
\newcommand{\Vvec}{\vec{V}}
\newcommand{\Wvec}{\vec{W}}
\newcommand{\Xvec}{\vec{X}}
\newcommand{\Yvec}{\vec{Y}}
\newcommand{\Zvec}{\vec{Z}}

\newcommand{\wvechat}{\hat{\vec{w}}}

\newcommand{\alphavec}{\vec{\alpha}}
\newcommand{\alphavecbar}{\bar{\vec{\alpha}}}
\newcommand{\alphavechat}{\hat{\vec{\alpha}}}
\newcommand{\alphavectil}{\tilde{\vec{\alpha}}}
\newcommand{\betavec}{\vec{\beta}}
\newcommand{\deltavec}{\vec{\delta}}
\newcommand{\etavec}{\vec{\eta}}
\newcommand{\phivec}{\vec{\phi}}
\newcommand{\psivec}{\vec{\psi}}
\newcommand{\thetavec}{\vec{\theta}}
\newcommand{\muvec}{\vec{\mu}}
\newcommand{\xivec}{\vec{\xi}}



\newcommand{\va}{{\tt a}}
\newcommand{\vc}{{\tt c}}
\newcommand{\vg}{{\tt g}}
\newcommand{\vp}{{\tt p}}
\newcommand{\vq}{{\tt q}}
\newcommand{\vs}{{\tt s}}
\newcommand{\vt}{{\tt t}}
\newcommand{\vu}{{\tt u}}
\newcommand{\vv}{{\tt v}}
\newcommand{\vw}{{\tt w}}
\newcommand{\vx}{{\tt x}}
\newcommand{\vy}{{\tt y}}
\newcommand{\val}{{\tt v}}


\newcommand{\Amat}{\Ab}
\newcommand{\Bmat}{\Bb}
\newcommand{\Cmat}{\Cb}
\newcommand{\Dmat}{\Db}
\newcommand{\Emat}{\Eb}
\newcommand{\Fmat}{\Fb}
\newcommand{\Gmat}{\Gb}
\newcommand{\Hmat}{\Hb}
\newcommand{\Imat}{\Ib}

\newcommand{\Vmat}{\Vb}
\newcommand{\Wmat}{\Wb}
\newcommand{\Xmat}{\Xb}
\newcommand{\Ymat}{\Yb}
\newcommand{\Zmat}{\Zb}


%%%%%%%% Greek letters definitions %%%%%%%%%%%%%%%
\newcommand{\delhat}{\hat{\delta}}
\newcommand{\delbar}{\bar{\delta}}
\newcommand{\ve}{\varepsilon}
\newcommand{\epsilonbar}{\bar{\epsilon}}
\newcommand{\etabar}{\bar{\eta}}
\newcommand{\Omegahat}{\hat{\Omega}}

\newcommand{\alphab}{\boldsymbol{\alpha}}
\newcommand{\betab}{\boldsymbol{\beta}}
\newcommand{\gammab}{\boldsymbol{\gamma}}
\newcommand{\Gammab}{\boldsymbol{\Gamma}}
\newcommand{\thetab}{\boldsymbol{\theta}}
\newcommand{\mub}{\boldsymbol{\mu}}
\newcommand{\phib}{\boldsymbol{\phi}}
\newcommand{\Deltab}{\boldsymbol{\Delta}}
\newcommand{\xib}{\boldsymbol{\xi}}
\newcommand{\lambdab}{\boldsymbol{\lambda}}
\newcommand{\Pib}{\boldsymbol{\Pi}}
\newcommand{\etab}{\boldsymbol{\eta}}
\newcommand{\taub}{\boldsymbol{\tau}}
\newcommand{\rhob}{\boldsymbol{\rho}}
\newcommand{\sigmab}{\boldsymbol{\sigma}}
\newcommand{\Sigmab}{\boldsymbol{\Sigma}}

\newcommand{\thetatil}{\tilde{\theta}}
\newcommand{\pitil}{\tilde{\pi}}


%%%%%%%% Mathematical definitions %%%%%%%%%%%%%%%

\newcommand{\CC}{\mathbb{C}} % Complex numbers
\newcommand{\EE}{\mathbb{E}} % Expectation
\newcommand{\HH}{\mathbb{H}} % Arbitrary field
\newcommand{\II}{\mathbb{I}} % Delta Indicator
\newcommand{\VV}{\mathbb{V}} % Variance
\newcommand{\KK}{\mathbb{K}} % Arbitrary field
\newcommand{\MM}{\mathbb{M}} % Median
\newcommand{\NN}{\mathbb{N}} % Natural numbers
\newcommand{\PP}{\mathbb{P}} % Probability
\newcommand{\QQ}{\mathbb{Q}} % Rationals
\newcommand{\RR}{\mathbb{R}} % Real numbers
\newcommand{\RRbar}{\overline{\mathbb{R}}} % Real numbers
\newcommand{\ZZ}{\mathbb{Z}} % Integers
\newcommand{\XX}{\mathbb{X}} % Integers
\newcommand{\YY}{\mathbb{Y}} % Integers

%%%%%%%%%%%%%%%  Mathematical Constants  %%%%%%%%%%%%%%%
\newcommand{\one}{\mathbf{1}}  % Identity
\newcommand{\zero}{\mathbf{0}} % Zero
\newcommand{\vone}{\mathbf{1}}  % Identity
\newcommand{\vzero}{\mathbf{0}} % Zero
\newcommand{\ind}{\boldsymbol{\mathsf{I}}}
\newcommand{\half}{\frac{1}{2}}
\newcommand{\sqrttwo}{\sqrt{2}}
\newcommand{\invsqrttwo}{\frac{1}{\sqrt{2}}}
\newcommand{\invtwo}{\frac{1}{2}}





%\newcommand{\TRUE}{\mathbf{TRUE}}  % True
%\newcommand{\FALSE}{\mathbf{FALSE}}  % False


%%%%%%%%%%%%%  Mathematical Operators  %%%%%%%%%%%%%%%

\DeclareMathOperator*{\mini}{\mathrm{minimize}}
\DeclareMathOperator*{\maxi}{\mathrm{maximize}}
\DeclareMathOperator*{\argmin}{\mathrm{argmin}}
\DeclareMathOperator*{\argmax}{\mathrm{argmax}}
\DeclareMathOperator*{\arginf}{\mathrm{arginf}}
\DeclareMathOperator*{\argsup}{\mathrm{argsup}}



%\newcommand*{\argmin}{\mathop{\mathrm{argmin}}}
%\newcommand*{\argmax}{\mathop{\mathrm{argmax}}}
%\newcommand*{\arginf}{\mathop{\mathrm{arginf}}}
%\newcommand*{\argsup}{\mathop{\mathrm{argsup}}}
%\newcommand{\sgn}{\mathop{\mathrm{sign}}}
%\newcommand{\tr}{\mathop{\mathrm{tr}}}
\newcommand{\trace}{\mathop{\mathrm{tr}}}
\newcommand{\diag}{\mathop{\mathrm{diag}}}
%\newcommand{\rank}{\mathop{\mathrm{rank}}}
\newcommand{\opvec}{\mathop{\mathrm{vec}}}
%\newcommand{\traj}{\mathop{\mathrm{Traj}}}
\newcommand*{\cov}{\mathrm{Cov}}
\newcommand*{\covars}{\mathrm{Cov}}
\newcommand*{\conv}{\mathrm{conv}}
\newcommand*{\const}{\mathrm{constant}}
%\newcommand{\tDiag}{\textsf{Diag}}
%\newcommand{\tTr}{\textsf{Tr}}
%\newcommand{\tE}{\textsf{E}}
%\newcommand{\tVec}{\textsf{Vec}}
%\newcommand{\tRank}{\textsf{Rank}}
\DeclareMathOperator{\sgn}{\mathop{\mathrm{sign}}}
\DeclareMathOperator{\tr}{\mathop{\mathrm{tr}}}
\DeclareMathOperator{\rank}{\mathop{\mathrm{rank}}}
\DeclareMathOperator{\traj}{\mathop{\mathrm{Traj}}}
\providecommand{\tra}{\intercal}
\newcommand{\ra}{\rightarrow}
\newcommand{\totvar}{\mathrm{TV}}


%%%%%%%% Thorems, Corollaries, etc %%%%%%%%%%%%%%%

%% Some style files might actually define these variables.
%% So don't mess with them if they are already defined

\ifx\BlackBox\undefined
\newcommand{\BlackBox}{\rule{1.5ex}{1.5ex}}  % end of proof
\fi

\ifx\QED\undefined
\def\QED{~\rule[-1pt]{5pt}{5pt}\par\medskip}
\fi

\ifx\proof\undefined
\newenvironment{proof}{\par\noindent{\bf Proof\ }}{\hfill\BlackBox\\[2mm]}
%\newenvironment{proof}{\emph{Proof. }}{ \hfill \QED}
\fi

\ifx\theorem\undefined
\newtheorem{theorem}{Theorem}
\newtheorem{example}{Example}
\newtheorem{property}{Property}
\newtheorem{lemma}[theorem]{Lemma}
\newtheorem{proposition}[theorem]{Proposition}
\newtheorem{remark}[theorem]{Remark}
\newtheorem{corollary}[theorem]{Corollary}
\newtheorem{definition}[theorem]{Definition}
\newtheorem{conjecture}[theorem]{Conjecture}
\newtheorem{assumption}[theorem]{Assumption}
\fi

\ifx\axiom\undefined
\newtheorem{axiom}[theorem]{Axiom}
\fi

%%%%%%%% Utility functions %%%%%%%%%%%%%%%

\newcommand{\eq}[1]{(\ref{#1})}
\newcommand{\mymatrix}[2]{\left[\begin{array}{#1} #2 \end{array}\right]}
\newcommand{\mychoose}[2]{\left(\begin{array}{c} #1 \\ #2 \end{array}\right)}
\newcommand{\mydet}[1]{\det\left[ #1 \right]}
\newcommand{\sembrack}[1]{[\![#1]\!]}
\newcommand{\mat}[1]{\mathbf{#1}}

\newcommand{\ea}{\emph{et al.}}
\newcommand{\eg}{\emph{e.g.}}
\newcommand{\ie}{\emph{i.e.}}
\newcommand{\iid}{\emph{iid}}
\newcommand{\etc}{\emph{etc.}}


%%%%%%%% Specific symbols for various projects %%%%%%%%%%%%%%%

\newcommand{\methodname}{KDE}
\newcommand{\hsic}{\mathrm{HSIC}}
\newcommand{\mmd}{\mathrm{MMD}}



%%%%%%%%%%%%%%%%%%%%%%%%%%%%%%%%%%%%%%%%%%%%%%%%%%%%%%%%%%%%%%%%%%%%%%%%%%%
% math symbols and commands
%%%%%%%%%%%%%%%%%%%%%%%%%%%%%%%%%%%%%%%%%%%%%%%%%%%%%%%%%%%%%%%%%%%%%%%%%%%
%\newcommand{\eq}[1]{(\ref{#1})}
%\newcommand{\mymatrix}[2]{\left[\begin{array}{#1} #2 \end{array}\right]}

%brackets
\newcommand{\inner}[2]{\left\langle #1,#2 \right\rangle}
\newcommand{\smallinner}[2]{{#1}^{\!\top}\!{#2}}  % no kernels allowed
\newcommand{\rbr}[1]{\left(#1\right)}
\newcommand{\sbr}[1]{\left[#1\right]}
\newcommand{\cbr}[1]{\left\{#1\right\}}
\newcommand{\nbr}[1]{\left\|#1\right\|}
\newcommand{\nbrsub}[2]{\left\|#1\right\|_{#2}}
\newcommand{\norm}[1]{\left\|#1\right\|}
\newcommand{\abr}[1]{\left|#1\right|}
\newcommand{\floor}[1]{\left\lfloor #1 \right\rfloor}
\newcommand{\ceil}[1]{\left\lceil #1 \right\rceil}
\newcommand{\smallfrac}[2]{{\textstyle \frac{#1}{#2}}}
\renewcommand{\url}[1]{{\sffamily #1}}
\newcommand{\arow}[2]{#1_{#2\cdot}}
\newcommand{\acol}[2]{#1_{\cdot#2}}
\newcommand{\twoco}[1]{\multicolumn{2}{c|}{#1}}
%\newcommand{\grad}[2]{\frac{\partial #1}{\partial #2}}
\newcommand{\pwrt}[1]{\frac{\partial}{\partial #1}}
\newcommand{\ppwrt}[1]{\frac{\partial^2}{(\partial #1)^2}}
\newcommand{\expunder}[1]{\mathop{\EE}\limits_{#1}}
\newcommand{\mean}[1]{\mathop{\EE}\sbr{{#1}}}
\newcommand{\vars}[2]{\var_{{#1}}\sbr{{#2}}}
\newcommand{\expec}[2]{\mathop{\EE}_{{#1}}\sbr{{#2}}}

%%%%%%%%%%%%%%%%%%%%%%  Algorithmic  %%%%%%%%%%%%%%%%%%%%%%
\renewcommand{\algorithmicrequire}{\textbf{Input:}}
\renewcommand{\algorithmicensure}{\textbf{Output:}}


\newcommand{\INPUT}{\STATE {\bfseries input}~}
\newcommand{\OUTPUT}{\STATE {\bfseries output}~}
\newcommand{\new}[1]{{\color{blue}\underline{\color{black}#1}}}
\newcommand{\comment}[1]{}


%%%%%%%%%%%%%%%%%%%%%%  Convex analysis related  %%%%%%%%%%%%%%%%%%%%%%

\newcommand{\dom}{\mathrm{dom}\,}
\newcommand{\myspan}[1]{\mathrm{span}\cbr{#1}}
\newcommand{\ri}{\mathrm{ri}\,}
\newcommand{\cl}{\mathrm{cl}\,}
\newcommand{\gstar}{g^{\star}}
\newcommand{\fstar}{f^{\star}}
\newcommand{\hstar}{h^{\star}}
\newcommand{\Astar}{A^{\star}}
\newcommand{\Kstar}{K^{\star}}
\newcommand{\Fstar}{F^{\star}}
\newcommand{\Gstar}{G^{\star}}
\def\infconv{\phantom{.}_{\vee}\negthickspace\!\!^{+}}
\newcommand{\gr}{{\mathrm{gr}}}
\newcommand{\epi}{{\mathrm{epi}}}
\newcommand{\inte}{\mathrm{int}\,}
\newcommand{\aff}{\mathrm{aff}\,}
\newcommand{\cone}{\mathrm{cone}\,}


%%%%%%%%%%%%%%%%%%%%%%  Graphical model related  %%%%%%%%%%%%%%%%%%%%%%
\def\ci{\perp\!\!\!\perp}
\newcommand{\given}{\mathrm{|}}
\newcommand{\MRF}{\mathrm{MRF}}


%%%%%%%%%%%%%%%%%%%%%%  Distribution related  %%%%%%%%%%%%%%%%%%%%%%
\newcommand{\Dir}{\mathrm{Dir}}
\newcommand{\dirp}{\mathrm{DP}}
\newcommand{\multi}{\mathrm{Multi}}
\newcommand{\gam}{\mathrm{Gamma}}
\newcommand{\var}{\mathrm{Var}}
\newcommand{\cor}{\mathrm{Corr}}
\newcommand{\bet}{\mathrm{Beta}}
\newcommand{\convdist}{\,\mathop \rightarrow \limits^{\Dcal}\,}
\newcommand{\convprob}{\,\mathop \rightarrow \limits^{\mathbb{P}}\,}
\newcommand{\convlone}{\,\mathop \rightarrow \limits^{L_1}\,}
\newcommand{\convlp}{\,\mathop \rightarrow \limits^{L_p}\,}
\newcommand{\convas}{\,\mathop \rightarrow \limits^{a.s.}\,}
\newcommand{\KL}{\mathrm{KL}}
\newcommand{\pdf}{\emph{pdf}}
\newcommand{\mrv}{\emph{mrv}}
\newcommand{\rv}{\emph{rv}}


%%%%%%%%%%%%%%%%%%%%%%  Kernel related  %%%%%%%%%%%%%%%%%%%%%%
\newcommand{\rmd}{\ensuremath{\mathrm{d}}}
\newcommand{\HSIC}{\ensuremath{\mathrm{HSIC}}}
\newcommand{\HSICb}{\ensuremath{\widehat{\mathrm{HSIC}}_b}}
\newcommand{\HSICu}{\ensuremath{\widehat{\mathrm{HSIC}}_u}}
\newcommand{\MMD}{\ensuremath{\mathrm{MMD}}}
\newcommand{\convhil}[1]{\,\mathop \rightarrow \limits^{#1}\,}
\newcommand{\convuni}{\rightrightarrows}


%%%%%%%%%%%%%%%%%%%%%%  BOMC related  %%%%%%%%%%%%%%%%%%%%%%
\newcommand{\idf}{{\tt idf}}
\newcommand{\tp}{\ensuremath{\mathrm{tp}}}
\newcommand{\fn}{\ensuremath{\mathrm{fn}}}
\newcommand{\fp}{\ensuremath{\mathrm{fp}}}
\newcommand{\tpb}{\ensuremath{\overline{\text{tp}}}}
\newcommand{\fnb}{\ensuremath{\overline{\text{fn}}}}
\newcommand{\fpb}{\ensuremath{\overline{\text{fp}}}}
\newcommand{\pre}{\ensuremath{\mathrm{Pre}}}
\newcommand{\rec}{\ensuremath{\mathrm{Rec}}}
\newcommand{\fs}{\ensuremath{\mathrm{F-score}}}
\newcommand{\opt}{\ensuremath{\mathrm{opt}}}

\newcommand{\topics}{{\tt topics}}
\newcommand{\industries}{{\tt industries}}
\newcommand{\regions}{{\tt regions}}
\newcommand{\ip}{{\tt ip}}
\newcommand{\category}{{\tt category}}
\newcommand{\svmbb}{{\tt SVM\_Batch}}
\newcommand{\lasvm}{{\tt LaSVM}}
\newcommand{\larank}{{\tt LaRank}}
\newcommand{\svmstruct}{$\text{\tt SVM}^{\text{\tt Struct}}$}
\newcommand{\svmperf}{{\tt SVMPerf}}
\newcommand{\bomccmn}{{\tt BOMC\_CMN}}
\newcommand{\bomcsam}{{\tt BOMC\_Sample}}
\newcommand{\bomc}{{\tt BOMC}}
\newcommand{\pasagg}{{\tt PA}}
\newcommand{\paonline}{{\tt PA\_OnlineCV}}
\newcommand{\pabatch}{{\tt PA\_BatchCV}}
\newcommand{\bomcind}{{\tt BOMC\_IND}}
\newcommand{\bomcindcmn}{{\tt BOMC\_IND\_CMN}}
\newcommand{\bomcindsam}{{\tt BOMC\_IND\_Sample}}
\newcommand{\trueskill}{TrueSkill$^{\text{TM}}$}
\newcommand{\svmlight}{\ensuremath{\mathrm{SVM}^{\mathrm{light}}}}


%%%%%%%%%%%%%%%%%%%%%%  Nesterov related  %%%%%%%%%%%%%%%%%%%%%%
\newcommand{\grad}{{\nabla}}
\newcommand{\gradbar}{{\bar{\nabla}}}
\newcommand{\mbarr}{{\overline{M(r)}}}
\newcommand{\card}{{\mathrm{card }}}
\newcommand{\MED}{{\text{MED}}}
\newcommand{\mubar}{\bar{\mu}}
\newcommand{\emp}{{\mathrm{emp}}}
\newcommand{\core}{{\mathrm{core}}}
\newcommand{\zerob}{{\mathbf{0}}}
\newcommand{\co}{{\mathrm{co}}}
\newcommand{\intr}{{\mathrm{int}}}
\newcommand{\bd}{{\mathrm{bd}}}
\newcommand{\BMRM}{{\sf BMRM}}
\newcommand{\bmrm}{{\sf BMRM}}
\newcommand{\where}{{\quad \text{where} \quad}}
\newcommand{\lcg}{{\textit{l.c.g}}}
\newcommand{\strcvx}{{\textit{s.c.}}}
\newcommand{\gradtil}{{\tilde{\grad}}}
\newcommand{\cp}{\mathrm{cp}}
\newcommand{\liblinear}{{\sf liblinear}}
\newcommand{\lsbmrm}{{\sf ls-bmrm}}
\newcommand{\qpbmrm}{{\sf qp-bmrm}}
\newcommand{\pegasos}{{\sf pegasos}}
\newcommand{\pegan}{{\sf pegasos-$n$}}
\newcommand{\pegaone}{{\sf pegasos-1}}
\newcommand{\nest}{{\sf pragam}}
\newcommand{\nestb}{{\sf pragam-b}}
\newcommand{\err}{{\sf err}}
\newcommand{\Xtr}{X_{\text{train}}}
\newcommand{\Xnew}{X_{\text{rep}}}
\newcommand{\Ctr}{{C_{\text{train}}}}
\newcommand{\yvectr}{{\yvec_{\text{train}}}}
\newcommand{\wcrf}{{\wvec_{\text{CRF}}}}
\newcommand{\wnode}{{\wvec_{\text{node}}}}
\newcommand{\wedges}{{\wvec_{\text{edge}}}}



\lecturenumber{1}       % assignment number
\duedate{12 noon, Feb 27, 2020}

% Fill in your name and email address
\stuinfo{}{}


\begin{document}

\maketitle

\section{Conditional Random Fields}

Suppose the training set consists of $n$ words.
The image of the $t$-th word can be represented as
$X^t = (\xvec^t_1, \ldots, \xvec^t_m)'$,
where $'$ means transpose,
$t$ is a superscript (not exponent),
and each \emph{row} of $X^t$ (\eg\ $\xvec^t_m$) represents a letter.
Here $m$ is the number of letters in the word,
and $\xvec^t_j$ is a 128 dimensional vector that represents its $j$-th letter image.
To ease notation, we simply assume all words have $m$ letters,
and the model extends naturally to the general case where the length of word varies.
The sequence label of a word is encoded as
$\yvec^t = (y^t_1, \ldots, y^t_m)$,
where $y^t_k \in \Ycal := \{1, 2, \ldots, 26\}$ represents the label of the $k$-th letter.
So in Figure \ref{fig:brace}, $y^t_1 = 2$, $y^t_2 = 18$, \ldots, $y^t_5 = 5$.

Using this notation, the Conditional Random Field (CRF) model for this task is a sequence shown in Figure \ref{fig:CRF_model},
and the probabilistic model for a word/label pair $(X, \yvec)$ can be written as
\begin{align}
	\label{eq:crf}
	p(\yvec | X ) &= \frac{1}{Z_X} \exp \rbr{\sum_{s=1}^m \inner{\wvec_{y_s}}{\xvec_s} + \sum_{s=1}^{m-1} T_{y_s, y_{s+1}}} \\
	\where Z_X &= \sum_{\hat{\yvec} \in \Ycal^m} \exp \rbr{\sum_{s=1}^m \inner{\wvec_{\yhat_s}}{\xvec_s} + \sum_{s=1}^{m-1} T_{\yhat_s, \yhat_{s+1}}}.
\end{align}
%
%$Z(X)$ is a normalization constant depending on $X$.
$\inner{\cdot}{\cdot}$ denotes inner product between vectors.
Two groups of parameters are used here:

\vspace{-1em}
\begin{itemize}
	\item {\bf Node weight:} Letter-wise discriminant weight vector $\wvec_k \in \RR^{128}$ for each possible letter label $k \in \Ycal$;
	\item {\bf Edge weight:} Transition weight matrix $T$ which is sized $26$-by-$26$.
	$T_{ij}$ is the weight associated with the letter pair of the $i$-th and $j$-th letter in the alphabet.  For example $T_{1,9}$ is the weight for pair (`a', `i'), and $T_{24,2}$ is for the pair (`x', `b'). In general $T$ is not symmetric, \ie\ $T_{ij} \neq T_{ji}$, or written as $T' \neq T$ where $T'$ is the transpose of $T$.
\end{itemize}

Given these parameters (\eg\ by learning from data), the model \eqref{eq:crf} can be used to predict the sequence label (\ie\ word) for a new word image $X^* := (\xvec^*_1, \ldots, \xvec^*_m)'$ via the so-called maximum a-posteriori (MAP) inference:
\begin{align}
	\label{eq:crf_decode}
	\yvec^* = \argmax_{\yvec \in \Ycal^m} p(\yvec | X^*)
	= \argmax_{\yvec \in \Ycal^m} \cbr{ \sum_{j=1}^m \inner{\wvec_{y_j}}{\xvec^*_j} + \sum_{j=1}^{m-1} T_{y_j, y_{j+1}}}.
\end{align}




\begin{itemize}
	\item[(1a)] {\bf [5 Marks]} Show that $\grad_{\wvec_y} \log p(\yvec|X)$---the gradient of $\log p(\yvec|X)$ with respect to $\wvec_y$---can be written as:
	\begin{align}
		\grad_{\wvec_y} \log p(\yvec^t|X^t) &= \sum_{s=1}^m (\sembrack{y^t_s = y} - p(y_s = y | X^t)) \xvec^t_s,
	\end{align}
	where $\llbracket \cdot \rrbracket = 1$ if $\cdot$ is true, and 0 otherwise.
	Show your derivation step by step.
	
	Now derive the similar expression for $\grad_{T_{ij}} \log p(\yvec|X)$.
	
		{\bf [Answer:]} 
	(i) $\grad_{\wvec_y} \log p(\yvec^t|X^t)$
	\begin{align}
		\grad_{\wvec_y} \log p(\yvec^t|X^t) &= \grad_{\wvec_y} \rbr{-logZ_{X^t} + \sum_{s=1}^m \inner{\wvec_{{y_s}^t}}{\xvec_s^t} + \sum_{s=1}^{m-1} T_{{y_s}^t, {y_{s+1}}^t}} \\ 
		&=\grad_{\wvec_y} \rbr{-logZ_{X^t} + \sum_{s=1}^m \inner{\wvec_{{y_s}^t}}{\xvec_s^t}}
	\end{align}

	First, we take gradient of the second term: 
	\begin{align}
		\grad_{\wvec_y} \sum_{s=1}^m \inner{\wvec_{{y_s}^t}}{\xvec_s^t} &= \sum_{s=1}^m \grad_{\wvec_y} (\wvec_{y_s^t}^T {{\xvec_s}^t}) \\
		&= \sum_{s=1}^m \sembrack{y^t_s = y} \xvec^t_s
	\end{align}

	Now, we take the gradient of the first term: 
	\begin{align}
		\label{eq:gradient_wy}
		-\grad_{\wvec_y} logZ_{X^t} &= -\frac{1}{Z_{X^t}} \sum_{\yvec \in \Ycal^m}\exp \rbr{\sum_{s=1}^m \inner{\wvec_{y_s}}{\xvec_s^t} + \sum_{s=1}^{m-1} T_{y_s, y_{s+1}}} \grad_{\wvec_y} \sum_{s=1}^m \inner{\wvec_{y_s}}{\xvec_s^t} \\
		&= -\sum_{\yvec \in \Ycal^m}p(\yvec|X^t) \sum_{s=1}^m \sembrack{y_s = y} \xvec^t_s \\
		&= -\sum_{s=1}^m p(y_s = y|X^t)\xvec^t_s
	\end{align}

	Therefore, we get: 
	\begin{align}
		\grad_{\wvec_y} \log p(\yvec^t|X^t) &= \sum_{s=1}^m (\sembrack{y^t_s = y} - p(y_s = y | X^t)) \xvec^t_s
	\end{align}
	\begin{flushright}
	Q.E.D.
	\end{flushright}

	(ii) $\grad_{T_{ij}} \log p(\yvec^t|X^t)$
	\begin{align}
		\grad_{T_{ij}} \log p(\yvec^t|X^t) &= \grad_{T_{ij}} \rbr{-logZ_{X^t} + \sum_{s=1}^m \inner{\wvec_{{y_s}^t}}{\xvec_s^t} + \sum_{s=1}^{m-1} T_{{y_s}^t, {y_{s+1}}^t}} \\ 
		&=\grad_{T_{ij}} \rbr{-logZ_{X^t} + \sum_{s=1}^{m-1} T_{{y_s}^t, {y_{s+1}}^t}}
	\end{align}

 	First, we take gradient of the second term: 
	\begin{align}
		\grad_{T_{ij}} \sum_{s=1}^{m-1} T_{{y_s}^t, {y_{s+1}}^t} &= \sum_{s=1}^{m-1} \grad_{T_{ij}} T_{{y_s}^t, {y_{s+1}}^t} \\
		&= \sum_{s=1}^{m-1} \sembrack{{y^t_s} = i, y^t_{s+1} = j}
	\end{align}

 	Now, we take the gradient of the first term: 
	\begin{align}
		\label{eq:gradient_T}
		-\grad_{T_{ij}} logZ_{X^t} &= -\frac{1}{Z_{X^t}} \sum_{\yvec \in \Ycal^m}\exp \rbr{\sum_{s=1}^m \inner{\wvec_{y_s}}{\xvec_s^t} + \sum_{s=1}^{m-1} T_{y_s, y_{s+1}}} \grad_{T_{ij}} \sum_{s=1}^{m-1} T_{{y_s}, {y_{s+1}}} \\
		&= -\sum_{\yvec \in \Ycal^m}p(\yvec|X^t) \sum_{s=1}^{m-1} \sembrack{y_s = i, y_{s+1} = j} \\
		&= -\sum_{s=1}^{m-1} p(y_s = i, y_{s+1} = j|X^t)
	\end{align}

	Therefore, we get: 
	\begin{align}
		\grad_{T_{ij}} \log p(\yvec^t|X^t) &= \sum_{s=1}^{m-1} (\sembrack{y^t_s = i, y^t_{s+1} = j} -  p(y_s = i, y_{s+1} = j|X^t))
	\end{align}
	Note that in the above notations, $y^t_s$ are known labels that are given, while $y_s$ is random variable. 

	\item[(1b)] {\bf [5 Marks]} A feature is a function that depends on $X$ and $\yvec$, but not $p(X|\yvec)$. Show that the gradient of $\log Z_X$ with respect to $\wvec_y$ and $T$ is exactly the expectation of some features with respect to $p(\yvec | X)$, and what are the features? Include your derivation.
	
	Hint: $T_{y_j, y_{j+1}} = \sum_{p \in \Ycal, q \in \Ycal} T_{pq} \cdot \llbracket (y_j, y_{j+1}) = (p,q)  \rrbracket$.

	{\bf [Answer:]} 

	As we have shown in \eqref{eq:gradient_wy}, $\grad_{\wvec_y} logZ_{X^t} = \sum_{\yvec \in \Ycal^m}p(\yvec|X^t) \sum_{s=1}^m \sembrack{y_s = y} \xvec^t_s$. 
	\begin{align}
		\sum_{\yvec \in \Ycal^m}p(\yvec|X^t) \sum_{s=1}^m \sembrack{y_s = y} \xvec^t_s = \EE_{\yvec \sim p(\yvec|X^t; \thetavec)} \sum_{s=1}^m \sembrack{y_s = y} \xvec^t_s
	\end{align} 

	where $\thetavec = [\wvec'_1, \ldots, \wvec'_{26}, T_{1,1}, T_{2,1}, \ldots, T_{26, 1}, T_{1,2}, \ldots, T_{26, 2}, \ldots, T_{1, 26}, \ldots, T_{26, 26}]'$.

	The feature in this case would be $\phi(X,\yvec) = \sum_{s=1}^m \sembrack{y_s = y} \xvec^t_s$. 

	In (1a) \eqref{eq:gradient_T}, we also showed that $\grad_{T_{ij}} logZ_{X^t} = \sum_{\yvec \in \Ycal^m}p(\yvec|X^t) \sum_{s=1}^{m-1} \sembrack{y_s = i, y_{s+1} = j} $.
	\begin{align}
		\sum_{\yvec \in \Ycal^m}p(\yvec|X^t) \sum_{s=1}^{m-1} \sembrack{y_s = i, y_{s+1} = j} = \EE_{\yvec \sim p(\yvec|X^t; \thetavec)} \sum_{s=1}^{m-1} \sembrack{y_s = i, y_{s+1} = j} 
	\end{align}

	The feature is $\phi(X,\yvec) = \sum_{s=1}^{m-1} \sembrack{y_s = i, y_{s+1} = j}; \forall (i, j) \in \Ycal = \{1, 2, ..., 26\}$, which does not depend on $X$. 

	
	\item[(1c)] {\bf [20 Marks]} Implement the decoder \eqref{eq:crf_decode} with computational cost $O(m|\Ycal|^2)$.
	You may use the max-sum algorithm introduced in the course, or any simplified dynamic programming method that is customized to the simple sequence structure. 
	To keep it simple, you do not need to implement a full-fledged message passing algorithm.
	It is also fine to use the recursive functionality in the programming language.
	
	Also implement the brute-force solution by enumerating $\yvec \in \Ycal^m$, which costs $O(|\Ycal|^m)$ time.  Try small test cases to make sure your implementation of dynamic programming is correct.
	
	The project package includes a test case stored in \verb#data/decode_input.txt#.
	It has a single word with 100 letters ($\xvec_1, \ldots, \xvec_{100}$), $\wvec_y$, and $T$, stored as a column vector in the form of
	\begin{align}
		[\xvec'_1, \ldots, \xvec'_{100}, \wvec'_1, \ldots, \wvec'_{26}, T_{1,1}, T_{2,1}, \ldots, T_{26, 1}, T_{1,2}, \ldots, T_{26, 2}, \ldots, T_{1, 26}, \ldots, T_{26, 26}]'.
	\end{align}
	All $\xvec_i \in \RR^{128}$ and $\wvec_j \in \RR^{128}$.
	
	\begin{center}
		\fbox{\begin{minipage}{38em}
				Be careful when loading $T$.  It is NOT a symmetric matrix.  Some languages store matrices in a row-major order while some use column-major.
		\end{minipage}}
	\end{center}
	
	In your submission, create a folder \verb#result# and store the result of decoding (the optimal $\yvec^* \in \Ycal^{100}$ of \eqref{eq:crf_decode}) in \underline{\texttt{result/decode\_output.txt}}.
	It should have 100 lines,
	where the $i$-th line contains one integer in $\{1,\ldots,26\}$ representing $y^*_i$.
	In your report, provide the maximum objective value $\sum_{j=1}^m \inner{\wvec_{y_j}}{\xvec_j} + \sum_{j=1}^{m-1} T_{y_j, y_{j+1}}$ for this test case.
	If you are using your own dynamic programming algorithm (\ie\ not max-sum),
	give a brief description especially the formula of recursion.
\end{itemize}

%\newpage
%Denote $m_{y_0 \to y_1}(y_1) = 1$ for all $y_1 \in \Ycal$.
%
%Forward: for $i = 2, 3, ... m-1$: $m_{y_i \to y_{i+1}}(y_{i+1}) = \sum_{y_i} m_{y_{i-1} \to y_{i}} (y_i)\exp (\inner{\wvec_{y_{i}}}{\xvec_i}) \exp(T_{y_i,y_{i+1}})$.
%
%Compute $Z = \sum_{y_m} m_{y_{m-1} \to y_m}(y_m) \exp (\inner{\wvec_{y_{m}}}{\xvec_m})$.
%
%Denote $m_{y_{m+1} \to y_m}(y_m) = 1$ for all $y_m \in \Ycal$.
%
%Backward: for $i = m, m-1,... 2$: $m_{y_i \to y_{i-1}}(y_{i-1}) = \sum_{y_i} m_{y_{i+1} \to y_{i}} (y_i)\exp (\inner{\wvec_{y_{i}}}{\xvec_i}) \exp(T_{y_{i-1},y_{i}})$.
%
%Compute marginals on $y_i$ for $i=1,\ldots,m$:
%$p(y_i) \propto m_{y_{i-1} \to y_i}(y_i) m_{y_{i+1} \to y_i} (y_i) \exp (\inner{\wvec_{y_{i}}}{\xvec_i})$, followed by local normalization.
%
%Compute marginals on edges for $i=1,\ldots, m-1$:
%$$p(y_i, y_{i+1}) \propto
%m_{y_{i-1} \to y_i}(y_i) m_{y_{i+2} \to y_{i+1}} (y_{i+1})
%\exp (\inner{\wvec_{y_{i}}}{\xvec_i})
%\exp (\inner{\wvec_{y_{i+1}}}{\xvec_{i+1}})
%\exp (T_{y_i, y_{i+1}}).$$
%
%Gradient in $\log Z$ are feature expectations computed from $p(y_i)$ and $p(y_i, y_{i+1})$.
%\newpage

\section{Training Conditional Random Fields}

Finally, given a training set $\{X^t, \yvec^t\}_{t=1}^n$ ($n$ words),
we can estimate the parameters $\{\wvec_k : k \in \Ycal\}$ and $T$ by maximizing the likelihood of the conditional distribution in \eqref{eq:crf}, or equivalently
\begin{align}
	\label{eq:obj_MLE}
	\min_{\{\wvec_y\}, T} \ -\frac{C}{n}\sum_{i=1}^n \log p(\yvec^i | X^i) + \frac{1}{2} \sum_{y \in \Ycal} \nbr{\wvec_y}^2 + \frac{1}{2} \sum_{ij} T^2_{ij}.
\end{align}
Here $C > 0$ is a trade-off weight that balances log-likelihood and regularization.


\begin{itemize}
	\item[(2a)] {\bf [20 Marks]} Implement a dynamic programming algorithm to compute $\log p(\yvec^i | X^i)$ and its gradient.  Recall that the gradient is nothing but the expectation of features, and therefore it suffices to compute the marginal distribution of $y_j$ and $(y_j, y_{j+1})$. The underlying dynamic programming principle is common to the computation of $\log p(\yvec^i | X^i)$, its gradient, and the decoder of \eqref{eq:crf_decode}.
	
	For numerical robustness (overflow or underflow), the following trick\footnote{\href{https://en.wikipedia.org/wiki/LogSumExp}{\url{https://en.wikipedia.org/wiki/LogSumExp}}} is widely used when computing $\log \sum_i \exp(x_i)$ for a given array $\{x_i\}$.  If we naively compute and store $\exp(x_i)$ as intermediate results, underflow and overflow could often occur.  So we resort to computing an equivalent form $M + \log \sum_i \exp(x_i - M)$, where $M := \max_i x_i$.  This way, the numbers to be exponentiated are always non-positive (eliminating overflow), and one of them is 0 (hence underflow is not an issue).  Similar tricks can be used for computing $\exp(x_1) / \sum_i \exp(x_i)$, or its logarithm.
	
	To ensure your implementation is correct, it is recommended that the computed gradient be compared against the result of numerical differentiation (which is based on the objective value only).
	In Python, use \verb#scipy.optimize.check_grad#.
	In Matlab, use \verb#gradest.m# from the DERIVESTsuite\footnote{\href{http://www.mathworks.com.au/matlabcentral/fileexchange/13490-adaptive-robust-numerical-differentiation}{\url{http://www.mathworks.com.au/matlabcentral/fileexchange/13490-adaptive-robust-numerical-differentiation}}}.
	In general, it is a very good practice to use these tools to test the implementation of function evaluator.
	Since numerical differentiation is often computation intensive, you may want to design small test cases (\eg\ a single word with 4 letters, 4 randomly valued pixels, and 3 letters in the alphabet).
	
	The project package includes a (big) test case in \verb#data/model.txt#.
	It specifies a value of $\wvec_y$ and $T$ as a column vector (again $T \neq T'$):
	\begin{align}
		\label{eq:model_vec}
		[\wvec'_1, \ldots, \wvec'_{26}, T_{1,1}, T_{2,1}, \ldots, T_{26, 1}, T_{1,2}, \ldots, T_{26, 2}, \ldots, T_{1, 26}, \ldots, T_{26, 26}]'.
	\end{align}
	Compute the gradient $\frac{1}{n} \sum_{i=1}^n \grad_{\wvec_y} \log p(\yvec^i | X^i)$ and
	$\frac{1}{n} \sum_{i=1}^n \grad_{T} \log p(\yvec^i | X^i)$
	(\ie\ averaged over the training set provided in \verb#data/train.txt#) evaluated at this $\wvec_y$ and $T$.
	Store them in \underline{\texttt{result/gradient.txt}} as a column vector following the same order as in \eqref{eq:model_vec}.
	Pay good attention to column-major / row-major of your programming language when writing $T$.
	
	{\bf Provide} the value of $\frac{1}{n} \sum_{i=1}^n \log p(\yvec^i | X^i)$ for this case in your report.
	
	For your reference,
	in your instructor's Matlab implementation (65 lines),
	it takes 5 seconds to compute the gradient on the whole training set.
	Single core.
	
	
	\item[(2b)] {\bf [20 Marks]} We can now learn ($\{\wvec_y\}, T$) by solving the optimization problem in \eqref{eq:obj_MLE} based on the training examples in \verb#data/train.txt#.
	Set $C = 1000$.
	Typical off-the-shelf solvers rely on a routine which, given as input a feasible value of the optimization variables ($\wvec_y, T$), returns the objective value and gradient evaluated at that ($\wvec_y, T$). This routine is now ready from the above task.
	
	In Matlab, you can use \verb#fminunc# from the optimization toolbox. In Python, you can use \verb#fmin_l_bfgs_b#, \verb#fmin_bfgs#, or \verb#fmin_ncg# from \verb#scipy.optimize#.
	Although \verb#fmin_l_bfgs_b# is for constrained optimization while \eqref{eq:obj_MLE} has no constraint, one only needs to set the bound to $(-\inf, \inf)$.  Set the initial values of $\wvec_y$ and $T$ to zero.
	
	Optimization solvers usually involve a large number of parameters.
	Some default settings for Matlab and Python solvers are provided in \verb#code/ref_optimize.m# and \verb#code/ref_optimize.py# respectively,
	where comments are included on the meaning of the parameters and other heuristics.
	It also includes some pseudo-code of CRF objective/gradient,
	to be used by various solvers.
	Read it even if you do not use Matlab, because similar settings might be used in other languages.
	Feel free to tune the parameters of the solvers if you understand them.
	
	In your submission, include
	\begin{itemize}
		\item The optimal solution $\wvec_y$ and $T$.  Store them as \underline{\tt{result/solution.txt}}, in the format of \eqref{eq:model_vec}.
		%
		\item The predicted label for each letter in the test data \verb#data/test.txt#, using the decoder implemented in (1c).
		Store them in \underline{\tt{result/prediction.txt}},
		with each line having one integer in $\{1,\ldots, 26\}$ that represents the predicted label of a letter, in the same order as it appears in \verb#data/test.txt#.
	\end{itemize}
	In your report, provide the optimal objective value of \eqref{eq:obj_MLE} found by your solver.
\end{itemize}



\section{Benchmarking with Other Methods}

Now we can perform some benchmarking by comparing with two alternative approaches:
multi-class linear SVM on individual letters (SVM-MC),
and structured SVM (SVM-Struct).
SVM-MC treats each pair of \emph{letter} image and label as a training/test example.
We will use the LibLinear package%
\footnote{\href{http://www.csie.ntu.edu.tw/~cjlin/liblinear/}{http://www.csie.ntu.edu.tw/$\sim$cjlin/liblinear/}},
which provides both Matlab and Python wrappers.
In order to keep the comparison fair,
we will use linear kernels only (there are kernelized versions of CRF),
and for linear kernels LibLinear is much faster than the general-purpose package LibSVM%
\footnote{\href{http://www.csie.ntu.edu.tw/~cjlin/libsvm/}{http://www.csie.ntu.edu.tw/$\sim$cjlin/libsvm/}},


For SVM-Struct, we will use the off-the-shelf implementation from the $\text{SVM}^{\text{hmm}}$ package%
\footnote{\href{http://www.cs.cornell.edu/People/tj/svm_light/svm_hmm.html}{http://www.cs.cornell.edu/People/tj/svm\_light/svm\_hmm.html}},
where some parameters are inherited from the $\text{SVM}^{\text{Struct}}$ package%
\footnote{\href{http://www.cs.cornell.edu/people/tj/svm_light/svm_struct.html}{http://www.cs.cornell.edu/people/tj/svm\_light/svm\_struct.html}}.
No Matlab/Python wrapper for $\text{SVM}^{\text{hmm}}$ is available.
So write scripts in your favorite language to call the binary executables and to parse the results.


$\text{SVM}^{\text{hmm}}$ requires that the input data be stored in a different format.
This conversion has been done for you, and the resulting data files are \verb#data/train_struct.txt# and \verb#data/test_struct.txt#.
The meaning of the fields in each line is described in \verb#data/fields_struct.txt#.
%Letter-wise multi-class SVM can be considered as a special case of structured SVM,
%where each word consists of only one letter.




\begin{itemize}
	\item[(3a)] {\bf [10 Marks]} $\text{SVM}^{\text{hmm}}$ has a number of parameters related to modeling, such as \verb#-c#, \verb#-p#, \verb#-o#, \verb#--t#, and \verb#--e#.
	Use the default settings for all parameters except \verb#-c#,
	which serves the same role as $C$ in \eqref{eq:obj_MLE} for CRF.
	In the sequel, we will also refer to the $C$ in \eqref{eq:obj_MLE} as the \verb#-c# parameter.
	LibLinear, which is used for SVM-MC, also has this parameter.
	But note that different from $\text{SVM}^{\text{hmm}}$ and \eqref{eq:obj_MLE}, the objective function used by LibLinear does NOT divide $C$ by the number of training examples (\ie\ letters).
	Keep the default value of other parameters in LibLinear.
	%So when invoking LibLinear, the value following \verb#-c# should be manually divided by $n$.
	
	The performance measure can be a) accuracy on letter-wise prediction, \ie\ the percentage of correctly predicted letters on the whole test set%
	\footnote{This is different from computing the percentage of correctly predicted letters in each word, and then averaging over all words, which is the last line of console output of \textsf{svm\_hmm\_classify}.  Both measures, of course, make sense.
		You may use the letter-wise prediction that \textsf{svm\_hmm\_classify} writes to the file specified by the third input argument.},
	or b) word-wise prediction, \ie\ the percentage of words whose constituent letters are all predicted correctly.
	For multi-class problems, accuracy is more commonly used than error.
	
	For each of CRF, SVM-Struct, and SVM-MC,
	plot a curve in a separate figure where the $y$-axis is the letter-wise prediction accuracy on test data,
	and the $x$-axis is the value of \verb#-c# varied in a range that you find reasonable,
	\eg\ $\{1, 10, 100, 1000\}$.
	Theoretically, a small \verb#-c# value will ignore the training data and generalize poorly on test data.
	On the other hand, overly large \verb#-c# may lead to overfitting, and make optimization challenging (taking a lot of time to converge).
	
	\textbf{Hint}: to roughly find a reasonable range of \verb#-c#, a commonly used heuristic is to try on a small sub-sample of the data, and then apply it to a larger data set (be wary of normalization by the number of training example for LibLinear as mentioned above).
	
	What observation can be made on the result?
	
	\item[(3b)] {\bf [5 Marks]} Produce another three plots for word-wise prediction accuracy on test data.  What observation can be made on the result?
	%Note in SVM-MC, we used the trick of treating each letter as a word. Therefore some conversion is needed to compute the word-wise accuracy. So another file \verb#data/test_index.txt# is provided.  Each line of it corresponds to a letter in the test set, and it has the same number of lines as \verb#test_mc.txt#.  Each line has a single integer which is the word-id from the original data.  So two lines with the same integer means their corresponding letters belong to the same word.
\end{itemize}


\section{Robustness to Distortion}

An evil machine learner tampered with the training and test images by rotation and translation.
However, the labels (letter/word annotation) are still clean, and so
one could envisage that the structured models (CRF and SVM-Struct) will be less susceptible to such tempering.

In Matlab, you can use \verb#imtranslate# and \verb#imrotate# to rotate and translate an image, respectively.
The manuals are at

\href{https://www.mathworks.com/help/images/ref/imtranslate.html}{https://www.mathworks.com/help/images/ref/imtranslate.html}

\href{https://www.mathworks.com/help/images/ref/imrotate.html}{https://www.mathworks.com/help/images/ref/imrotate.html}

They both take some parameters such as offset and degree of rotation.
Make sure that the image size is not changed. 
This means in \verb#imtranslate#, 
do use \verb#'FillValues', 0# and do NOT use \verb#'OutputView','full'#.
By default, \verb#imtranslate# returns an image with the same size as the input.
For \verb#imrotate#, however, the default setting does change the image size, and you need to explicitly use the \verb#crop# option to keep the output image size unchanged.
You can choose any interpolation method.

Similarly, in Python, you could use functions from OpenCV such as
\texttt{getRotationMatrix2D} and \texttt{warpAffine} for rotation and
translation. More information is present \href{https://opencv-python-tutroals.readthedocs.io/en/latest/py\_tutorials/py\_imgproc/py\_geometric\_transformations/py\_geometric\_transformations.html}{here}.

If you do not use Matlab or Python, read the above two links and find the corresponding functions in your language.  
They are pretty standard functions.	

Both \verb#imtranslate# and \verb#imrotate# take images represented as a matrix.
So if your image is represented as a 128 dimensional vector,
first reshape it into a matrix by \verb#reshape(x, 8, 16)#,
then apply these functions,
followed by vectorizing it back.

The images stored in the data files are in column-major order:

\href{https://en.wikipedia.org/wiki/Row-_and_column-major_order}{https://en.wikipedia.org/wiki/Row-\_and\_column-major\_order}

Matlab uses column-major while Python uses row-major.
As you need to translate and rotate images here, 
you may want to ensure that the 2-D image is properly loaded before applying the transformations.  
Just check by visualizing the images, e.g. by \verb#imshow# in Matlab.	

In this experiment
we randomly select a subset of training examples to distort.
All test examples remain unchanged.
A randomly generated list of transformations are given in \verb#data/transform.txt#, where the lines are in the following format:
%
\begin{verbatim}
r 317 15
t 2149 3 3
\end{verbatim}
The first line means: on the 317-th word of the training data (in the order of \verb#train.txt#),
apply counterclockwise rotation by 15 degrees \emph{to all its letters}.
The second line means on the 2149-th word of the training data, apply translation with offset $(3,3)$.
Note in each line the first number (\ie\ second column: 317, 2149, \ldots) is random and \emph{not} sorted.
All line numbers appear exactly once.

\begin{itemize}
	\item[(4a)] {\bf [10 Marks]} In one figure, plot the following two curves where the $y$-axis is the letter-wise prediction accuracy on test data.  We will apply to the training data the first $x$ lines of transformations specified in \verb#data/transform.txt#.  $x$ is varied in $\{0, 500, 1000, 1500, 2000\}$ and serves as the value of $x$-axis.
	
	1) CRF where the \verb#-c# parameter is set to any of the best values found in (3a);
	
	2) SVM-MC where the \verb#-c# parameter is set to any of the best values found in (3a).
	
	What observation can be made on the result?
	
	\item[(4b)] {\bf [5 Marks]}  Generate another plot for word-wise prediction accuracy on test data.  The \verb#-c# parameter in SVM-MC may adopt any of the best values found in (3b).
	What observation can be made on the result?
	
\end{itemize}

\section{Appendix: Dynamic programming details}

\subsection{Computing the partition function}

In order to compute $Z$, let us define $f_1(y_1) = 1$ for all $y_1 \in \Ycal$, and then for all $i = 2, \ldots, m$
\begin{align}
	f_i(y_i) = \sum_{y_1, \ldots, y_{i-1}} \exp \rbr{\sum_{j=1}^{i-1} \inner{\wvec_{y_j}}{\xvec_j} + \sum_{j=1}^{i-1} T_{y_j, y_{j+1}}}.
\end{align}
This is exactly the message $m_{(i-1) \to i}(y_{i})$ defined in our lecture of Mar 1.
Then
\begin{align}
	Z = \sum_{y_m \in \Ycal} \exp \rbr{\inner{\wvec_{y_m}}{\xvec_m}} f_m(y_m) .
\end{align}

To compute $f_i(y_i)$, we use recursion by
\begin{align}
	\label{eq:rec_Z_forward_1}
	f_1(y_1) &= 1 \\
	f_i(y_i) &= \sum_{y_{i-1}} \exp \rbr{\inner{\wvec_{y_{i-1}}}{\xvec_{i-1}} + T_{y_{i-1}, y_i}}
	\sum_{y_1, \ldots, y_{i-2}} \exp \rbr{\sum_{j=1}^{i-2} \inner{\wvec_{y_j}}{\xvec_j} + \sum_{j=1}^{i-2} T_{y_j, y_{j+1}}} \\
	\label{eq:rec_Z_forward_2}
	&= \sum_{y_{i-1}} \exp \rbr{\inner{\wvec_{y_{i-1}}}{\xvec_{i-1}} + T_{y_{i-1}, y_i}}  \cdot f_{i-1}(y_{i-1})  \qquad (\forall\ i = 2, 3, \ldots, m).	 
\end{align}

\paragraph{Comment 1:}
We can also use the log-sum-exp trick if we stay in the logarithmic space. 
Define $\alpha_i(y_i) = \log f_i(y_i)$.
Then $\alpha_1(y_1) = 0$, and
\begin{align}
	\alpha_i(y_i) &= \log \sum_{y_{i-1}} \exp \rbr{\inner{\wvec_{y_{i-1}}}{\xvec_{i-1}} + T_{y_{i-1}, y_i} + \alpha_{i-1}(y_{i-1})} \qquad \forall i \ge 2\\
	\log Z &= \log \sum_{y_m} \exp (\inner{\wvec_{y_m}}{\xvec_m} + \alpha_m(y_m)).
\end{align}

\paragraph{Comment 2:}
We might attempt to incorporate the node factor $\exp(\inner{\wvec_{y_i}}{\xvec_i})$ into the message, i.e.,
\begin{align}
	g_i(y_i) := \exp(\inner{\wvec_{y_i}}{\xvec_i}) \cdot f_i(y_i) = \sum_{y_1, \ldots, y_{i-1}} \exp \rbr{\sum_{j=1}^{i} \inner{\wvec_{y_j}}{\xvec_j} + \sum_{j=1}^{i-1} T_{y_j, y_{j+1}}}.
\end{align}
Then the recursion can be written as
\begin{align}
	g_1(y_1) &= \exp(\inner{\wvec_{y_1}}{\xvec_1}) \\
	g_i(y_i) &= \exp(\inner{\wvec_{y_i}}{\xvec_i}) \cdot \sum_{y_{i-1}} \exp \rbr{T_{y_{i-1}, y_i}}
	\sum_{y_1, \ldots, y_{i-2}} \exp \rbr{\sum_{j=1}^{i-1} \inner{\wvec_{y_j}}{\xvec_j} + \sum_{j=1}^{i-2} T_{y_j, y_{j+1}}} \\
	&= \exp(\inner{\wvec_{y_i}}{\xvec_i}) \cdot \sum_{y_{i-1}} \exp \rbr{T_{y_{i-1}, y_i}}  g_{i-1}(y_{i-1})  \qquad (\forall\ i \ge 2), \\
	\text{and} \quad Z &= \sum_{y_m} f_m(y_m).
\end{align}
This works well for a linear chain.
But it is not recommended because it does not extend well to general tree structure.
Suppose we want to compute the message from $j$ to $i$, and $j$ has two other neighbors $k$ and $k'$.
Then the messages $k \to j$ and $k' \to j$ have \textbf{both} contained the node potential of $j$ (i.e. $f_j(y_j) = \exp (\inner{\wvec_{y_j}}{\xvec_j})$).
So when we multiply together the messages $k \to j$ and $k' \to j$ (as in the message formula),
this node potential will be \emph{double counted}.

\subsection{MAP inference}
Here we define
\begin{align}
	h_i(y_i) = \max_{y_1, \ldots, y_{i-1}} \exp \rbr{\sum_{j=1}^{i-1} \inner{\wvec_{y_j}}{\xvec_j} + \sum_{j=1}^{i-1} T_{y_j, y_{j+1}}}.
\end{align}

Then all the recursion expressions in \eqref{eq:rec_Z_forward_1}	to \eqref{eq:rec_Z_forward_2} almost remain unchanged,
except that all summations are replaced by max:
\begin{align}
	\label{eq:rec_MAP_forward_1}
	h_1(y_1) &= 1 \\
	h_i(y_i) &= \max_{y_{i-1}} \exp \rbr{\inner{\wvec_{y_{i-1}}}{\xvec_{i-1}} + T_{y_{i-1}, y_i}}
	\max_{y_1, \ldots, y_{i-2}} \exp \rbr{\sum_{j=1}^{i-2} \inner{\wvec_{y_j}}{\xvec_j} + \sum_{j=1}^{i-2} T_{y_j, y_{j+1}}} \\
	\label{eq:rec_MAP_forward_2}	
	&= \max_{y_{i-1}} \exp \rbr{\inner{\wvec_{y_{i-1}}}{\xvec_{i-1}} + T_{y_{i-1}, y_i}}  h_{i-1}(y_{i-1})  \qquad (\forall\ i = 2, 3, \ldots, m).
\end{align}

After obtaining $h_m(y_m)$, we recover the MAP by backtracking
\begin{align}
	y^*_m &= \argmax_{y_m} \cbr{ \exp(\inner{\wvec_{y_m}}{\xvec_m}) h_m(y_m)} \\
	y^*_{i-1} &= \argmax_{y_{i-1}} \exp \rbr{\inner{\wvec_{y_{i-1}}}{\xvec_{i-1}} + T_{y_{i-1}, y^*_i}}  h_{i-1}(y_{i-1})  \qquad (\forall\ i = m, m-1, \ldots, 2).
\end{align}

\paragraph{Comment 3:}
The above algorithm is called max-product.
We can also take the log of $h_i(y_i)$ and get the max-sum algorithm:
\begin{align}
	l_i(y_i) = \log h_i(y_i) = \max_{y_1, \ldots, y_{i-1}} \cbr{\sum_{j=1}^{i-1} \inner{\wvec_{y_j}}{\xvec_j} + \sum_{j=1}^{i-1} T_{y_j, y_{j+1}}}.
\end{align}

And the recursion goes by
\begin{align}
	l_1(y_1) &= 0 \\
	l_i(y_i) &= \max_{y_{i-1}} \cbr{\inner{\wvec_{y_{i-1}}}{\xvec_{i-1}} + T_{y_{i-1}, y_i} + l_{i-1}(y_{i-1})}.
\end{align}
And the recovery is
\begin{align}
	y^*_m &= \argmax_{y_m} \cbr{ \inner{\wvec_{y_m}}{\xvec_m} + l_m(y_m)} \\
	y^*_{i-1} &= \argmax_{y_{i-1}} \cbr{ \inner{\wvec_{y_{i-1}}}{\xvec_{i-1}} + T_{y_{i-1}, y^*_i} + l_{i-1}(y_{i-1})}  \qquad (\forall\ i = m, m-1, \ldots, 2).
\end{align}

\subsection{Marginal distribution}

To compute the marginal distribution, we need the backward messages.
Define $b_m(y_m) = 1$ for all $y_m \in \Ycal$ and then for all $i = m-1, \ldots, 1$
\begin{align}
	b_i(y_i) = \sum_{y_{i+1}, \ldots, y_{m}} \exp \rbr{\sum_{j=i+1}^m \inner{\wvec_{y_j}}{\xvec_j} + \sum_{j=i}^{m-1} T_{y_j, y_{j+1}}}.
\end{align}
This is exactly the message $m_{(i+1) \to i} (y_i)$ defined in our lecture.
Then the partition function can be computed by
\begin{align}
	Z = \sum_{y_1} \exp \rbr{\inner{\wvec_{y_1}}{\xvec_1}} b_1(y_1).
\end{align}


To compute $b_i(y_i)$, we use recursion by
\begin{align}
	\label{eq:rec_Z_forward_1}
	b_m(y_m) &= 1 \\
	b_i(y_i) &= \sum_{y_{i+1}} \exp \rbr{\inner{\wvec_{y_{i+1}}}{\xvec_{i+1}} + T_{y_{i}, y_{i+1}}}
	\sum_{y_{i+2}, \ldots, y_m} \exp \rbr{\sum_{j=i+2}^{m} \inner{\wvec_{y_j}}{\xvec_j} + \sum_{j=i+1}^{m-1} T_{y_j, y_{j+1}}} \\
	\label{eq:rec_Z_forward_2}
	&= \sum_{y_{i+1}}\exp \rbr{\inner{\wvec_{y_{i+1}}}{\xvec_{i+1}} + T_{y_{i}, y_{i+1}}} b_{i+1}(y_{i+1})  \qquad (\forall\ i = m-1, m-2, \ldots, 1).	 
\end{align}

Finally, the marginal distribution of $y_i$ is
\begin{align}
	p(y_i) \propto f_i(y_i) \cdot b_i(y_i) \cdot \exp \rbr{\inner{\wvec_{y_{i}}}{\xvec_{i}}},
\end{align}
followed by local normalization.
Furthermore, the marginal distribution of $(y_i, y_{i+1})$ is
\begin{align}
	p(y_i, y_{i+1}) \propto f_i(y_i) \cdot b_{i+1}(y_{i+1}) \cdot 
	\exp \rbr{\inner{\wvec_{y_{i}}}{\xvec_{i}} + \inner{\wvec_{y_{i+1}}}{\xvec_{i+1}} + T_{y_i, y_{i+1}}}.
\end{align}

\paragraph{Comment 4:}
For numerical robustness,
we can also turn the backward messages into log space, similar to Comment 1.




\end{document}
